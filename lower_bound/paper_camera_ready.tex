%\documentclass[10pt,onecolumn,a4paper]{article}
\documentclass{article} % For LaTeX2e
\usepackage{times}
\usepackage{url}
\usepackage{times}
\usepackage{epsfig}
\usepackage{graphicx}
\usepackage{amsmath, amsthm, amssymb}
\usepackage{algorithm}
\usepackage{algorithmic}
\usepackage{natbib}
\usepackage{bibentry}
\usepackage{fullpage}
%\usepackage{authblk}


\renewcommand{\Pr}{\field{P}}

\newcommand{\bx}{\boldsymbol{x}}
\newcommand{\bX}{\boldsymbol{X}}
\newcommand{\bu}{\boldsymbol{u}}
%\newcommand{\bu}{h}
\newcommand{\h}{h}
\newcommand{\by}{\boldsymbol{y}}
\newcommand{\bY}{\boldsymbol{Y}}
\newcommand{\bH}{\boldsymbol{H}}
\newcommand{\sY}{\mathcal{Y}}
\newcommand{\bhatY}{\boldsymbol{\hat{Y}}}
\newcommand{\bbary}{\boldsymbol{\bar{y}}}
\newcommand{\bg}{\boldsymbol{g}}
\newcommand{\bz}{\boldsymbol{z}}
%\newcommand{\bzt}{k(\bx_t,\cdot)}
\newcommand{\bzt}{k_t}
\newcommand{\bzi}{k_i}
\newcommand{\bZ}{\boldsymbol{Z}}
\newcommand{\bbarZ}{\boldsymbol{\bar{Z}}}
\newcommand{\bbarz}{\boldsymbol{\bar{z}}}
\newcommand{\bhatZ}{\boldsymbol{\hat{Z}}}
\newcommand{\bhatz}{\boldsymbol{\hat{z}}}
\newcommand{\bbarw}{\bar{f}}
\newcommand{\barz}{\bar{z}}
\newcommand{\bS}{\boldsymbol{S}}
\newcommand{\bbarS}{\boldsymbol{\bar{S}}}
\newcommand{\bw}{\boldsymbol{w}}
%\newcommand{\bw}{f}
\newcommand{\f}{f}
\newcommand{\bW}{\boldsymbol{W}}
\newcommand{\bU}{\boldsymbol{U}}
\newcommand{\bv}{\boldsymbol{v}}
\newcommand{\bzero}{\boldsymbol{0}}
\newcommand{\balpha}{\boldsymbol{\alpha}}
\newcommand{\sA}{\mathcal{A}}
\newcommand{\sC}{\mathcal{C}}
\newcommand{\sX}{\mathcal{X}}
\newcommand{\sS}{\mathcal{S}}
\newcommand{\sZ}{\mathcal{Z}}
\newcommand{\sbarZ}{\bar{\mathcal{Z}}}
\newcommand{\fbag}{\bold{F}}


\newcommand{\inner}[1]{ \langle {#1} \rangle }

%\newcommand{\argmin}[1]{\underset{#1}{\operatorname{argmin}}}
\DeclareMathOperator*{\argmin}{arg\,min}
\DeclareMathOperator*{\argmax}{arg\,max}
%\newcommand{\argmax}[1]{\underset{#1}{\operatorname{argmax}}}

\newcommand{\todo}[1]{\textcolor{red}{TODO: #1}}
\newcommand{\fixme}[1]{\textcolor{red}{FIXME: #1}}

\newcommand{\field}[1]{\mathbb{#1}}
\newcommand{\set}[1]{\mathcal{#1}}
\newcommand{\fY}{\set{Y}}
\newcommand{\fX}{\set{X}}
\newcommand{\fH}{\set{H}}

\newcommand{\R}{\field{R}}
\newcommand{\Nat}{\field{N}}
\newcommand{\E}{\field{E}}
\newcommand{\Var}{\mathrm{Var}}

\newcommand{\X}{\mathcal{X}}
\newcommand{\Y}{\mathcal{Y}}


\newcommand{\btheta}{{\boldsymbol{\theta}}}
%\newcommand{\btheta}{g}
\newcommand{\bbartheta}{\boldsymbol{\bar{\theta}}}
\newcommand\theset[2]{ \left\{ {#1} \,:\, {#2} \right\} }
\newcommand\inn[2]{ \left\langle {#1} \,,\, {#2} \right\rangle }
\newcommand\innK[2]{ \left\langle {#1} \,,\, {#2} \right\rangle_K }
\newcommand\RE[2]{ D\left({#1} \| {#2}\right) }
\newcommand\Ind[1]{ \left\{{#1}\right\} }
\newcommand{\norm}[1]{\left\|{#1}\right\|}
\newcommand{\diag}[1]{\mbox{\rm diag}\!\left\{{#1}\right\}}

\newcommand{\defeq}{\stackrel{\rm def}{=}}
\newcommand{\sgn}{\mbox{\sc sgn}}
\newcommand{\scI}{\mathcal{I}}
\newcommand{\scO}{\mathcal{O}}

\newcommand{\dt}{\displaystyle}
\renewcommand{\ss}{\subseteq}
\newcommand{\wh}{\widehat}
\newcommand{\ve}{\varepsilon}
\newcommand{\hlambda}{\wh{\lambda}}
\newcommand{\yhat}{\wh{y}}

\newcommand{\hDelta}{\wh{\Delta}}
\newcommand{\hdelta}{\wh{\delta}}
\newcommand{\spin}{\{-1,+1\}}

\newcommand{\theHalgorithm}{\arabic{algorithm}}

\newtheorem{lemma}{Lemma}
\newtheorem{theorem}{Theorem}
\newtheorem{cor}{Corollary}
\newtheorem{definition}{Definition}
\newtheorem{assumption}{Assumption}
%\newtheorem{remark}{Remark}
%\newtheorem{prop}{Proposition}

\newcommand{\reals}{\mathbb{R}}
\newcommand{\sign}{{\rm sign}}


\newcommand{\q}{q}
\renewcommand{\ng}{\theta}

\newcommand{\G}{\mathcal{G}}
\newcommand{\W}{\mathcal{W}}
\newcommand{\opteps}{\eps^*}

\newcommand{\Risk}{\mathcal{R}}
\newcommand{\RiskLoss}{\mathcal{E}^{\ell}}
\newcommand{\Ltworho}{\mathcal{L}^2_{\rho_\fX}}
\newcommand{\Lonerho}{\mathcal{L}^1_{\rho_\fX}}
\newcommand{\frho}{f_\rho}
\newcommand{\flrho}{f^{\ell}_\rho}
\newcommand{\fcla}{f_c}
\newcommand{\fl}{f^{\ell}}
\newcommand{\flambda}{f_\lambda}
\newcommand{\HK}{\mathcal{H}_K}

%\newcommand{\normK}[1]{\left\|{#1}\right\|_{\mathcal{H}_K}}
\newcommand{\normK}[1]{\left\|{#1}\right\|_K}
\newcommand{\normL}[1]{\left\|{#1}\right\|_{\Ltworho}}
\newcommand{\normLOne}[1]{\left\|{#1}\right\|_{\Lonerho}}

\DeclareMathOperator\erf{erf}

\newcommand{\gain}{Reward}
\newcommand{\sg}{G}

\begin{document}

%%%%%%%%% TITLE
\title{Optimal Non-Asymptotic Lower Bound to the Minimax Regret in the Expert Setting}
%\author{Francesco Orabona }
%\author{David Pal  }
%\affil{Yahoo! Labs\\New York, NY, USA}
\author{
\begin{tabular}{c@{\hskip 1in}c}
  Francesco Orabona & David Pal \\
  francesco@orabona.com & dpal@yahoo-inc.com \\
\end{tabular}
\\\\
Yahoo Labs\\New York, NY, USA
}


\maketitle

\begin{abstract}%   <- trailing '%' for backward compatibility of .sty file
We prove non-asymptotic lower bounds to the expectation of the max of Gaussian and Binomial variables.
As an application, we prove a very simple non-asymptotic lower bound to the minimax regret of online learning in the experts scenario. The lower bound also recovers the optimal leading constant in the limit.
\end{abstract}

\section{Introduction}

It is known that the lower bound to the regret in the expert setting is $\Omega(\sqrt{n \log d})$. Two proofs are known, one non-asymptotic and one asymptotic, with the latter recovering the optimal leading constant~\citep{Cesa-BianchiL06}.

In this short report we show a simple non-asymptotic lower bound to expectation of Gaussian and Binomial variables and we use it to prove a lower bound that has the best of the best known bounds.

We will first prove the lower bound for the maximum of Gaussians. The Binomial case will closely follow it.
Then, we will show the lower bound on the minimax regret in the experts setting.

\section{Gaussian Case}

The Gaussian case will be a warm-up for the proof of the Binomial case.

It is well known that the maximum of Gaussian variables converges to a Gumbel distribution. Here we quantify the rate of convergence for the expectation of maximum of independent Gaussian variables.

We will make use of the following technical Lemma on the Mill's ratio of Gaussian variables.
\begin{lemma}[\cite{Boyd59}]
\label{lemma:boyd}
Let $x\geq0$, then
\[
\exp\left(\frac{x^2}{2}\right) \int_{x}^{+\infty} \exp\left(-\frac{t^2}{2}\right) dt
\geq \frac{\pi}{(\pi-1)x+\sqrt{x^2+2 \pi}}.
\]
\end{lemma}
%
\begin{cor}
Let $X \sim N(\mu, \sigma^2)$ and $t \geq 0$. Then
\[
P(X>t) \geq \exp\left(-\frac{(t-\mu)^2}{2 \sigma^2}\right) \frac{1}{\sqrt{2\pi}\frac{t-\mu}{\sigma}+2}.
\]
\end{cor}
%
\begin{proof}
\begin{align*}
P(X>t)
&= \frac{1}{\sigma \sqrt{2 \pi}} \int_t^{\infty} \exp(-\frac{(x-\mu)^2}{2 \sigma^2}) d x 
= \frac{1}{\sqrt{2 \pi}} \int_\frac{t-\mu}{\sigma}^{\infty} \exp(-\frac{x}{2}) d x \\
&\geq \frac{1}{\sqrt{2 \pi}} \exp\left(-\frac{(t-\mu)^2}{2 \sigma^2}\right) \frac{\pi}{(\pi-1)\frac{t-\mu}{\sigma}+\sqrt{\frac{(t-\mu)^2}{\sigma^2}+2 \pi}} \\
&\geq \frac{1}{\sqrt{2}} \exp\left(-\frac{(t-\mu)^2}{2 \sigma^2}\right) \frac{\sqrt{\pi}}{\pi\frac{t-\mu}{\sigma}+\sqrt{2 \pi}} \\
&= \exp\left(-\frac{(t-\mu)^2}{2 \sigma^2}\right) \frac{1}{\sqrt{2\pi}\frac{t-\mu}{\sigma}+2},
\end{align*}
where in the first inequality we used Lemma~\ref{lemma:boyd}.
\end{proof}


% \begin{theorem}
% \[
% E[\max_{i=1,\cdots,d} X_i] \geq \sigma 0.28 \sqrt{\log d} -\sqrt{\frac{2}{\pi}} \sigma
% \]
% \end{theorem}
% \begin{proof}
% Define the event $A$ equal to the case that at least one of the $X_i$ is greater or equal than $C \sigma \sqrt{\log d}$. 
% \begin{align}
% E[\max_{i=1,\cdots,d} X_i] 
% &= E[\max_{i=1,\cdots,d} X_i| A ] P(A) + E[\max_{i=1,\cdots,d} X_i| not A ] P( not A) \\
% &\geq E[\max_{i=1,\cdots,d} X_i| A ] P(A) + E[\max_{i=1,\cdots,d} X_i| not A ]\\
% &\geq E[\max_{i=1,\cdots,d} X_i| A ] P(A) + E[X_1| not A ]\\
% &= E[\max_{i=1,\cdots,d} X_i| A ] P(A) + E[X_1| X_1\leq C \sigma \sqrt{\log d}]\\
% &\geq E[\max_{i=1,\cdots,d} X_i| A ] P(A) + E[X_1| X_1\leq 0]\\
% &\geq E[\max_{i=1,\cdots,d} X_i| A ] P(A) - \sqrt{\frac{2}{\pi}} \sigma\\
% &\geq C \sigma \sqrt{\log d} P(A) - \sqrt{\frac{2}{\pi}} \sigma.
% \end{align}
% 
% Now let's focus on $P(A)$. It is equal to
% \begin{align}
% P(A) 
% &= 1-P(X_1\leq C \sigma \sqrt{\log d})^d \\
% &= 1-(1-P(X_1\geq C \sigma \sqrt{\log d}))^d \\
% &\geq 1-\left(1-\exp\left(-\frac{C^2 \log d}{2}\right) \frac{1}{\sqrt{2\pi}C \sqrt{\log d}+2}\right)^d \\
% &= 1-\left(1-\frac{d^{-\frac{C^2}{2}}}{\sqrt{2\pi}C \sqrt{\log d}+2}\right)^d \\
% &= 1- \exp\left(d \log\left(1-\frac{d^{-\frac{C^2}{2}}}{\sqrt{2\pi}C \sqrt{\log d}+2}\right)\right) \\
% &\geq 1 - \exp\left(-\frac{d^{1-\frac{C^2}{2}}}{\sqrt{2\pi}C \sqrt{\log d}+2}\right)\\
% &\geq 1 - \exp\left(-\frac{d^{1-\frac{C^2}{2}}}{\sqrt{\frac{2\pi}{e}}d^\frac{C^2}{2}+2}\right)\\
% \end{align}
% where in the second to last inequality we used the elementary inequality $\log(1-\frac{1}{x}) \leq -\frac{1}{x}, \forall  0\leq x<1$ and in the last inequality we used $\log(x) \leq \frac{1}{a e} x^a, \forall x,a>0$.
% We need this last expression to be increasing in $d$, that is we need $\frac{d^{1-\frac{C^2}{2}}}{\sqrt{\frac{2\pi}{e}}d^\frac{C^2}{2}+2}$ to be increasing in $d$.
% From the derivative, we have that this is true iff
% \[
% \sqrt(2 e \pi) (1-C^2) d^{\frac{C^2}{2}}+e (2-C^2) \geq 0, \forall d\geq2.
% \]
% Considering only $0 \leq  C\leq \sqrt{2}$, the stricter condition is with $d=2$ and numerically we have $C\leq 1.138$.
% So, we have
% \begin{align}
% P(A) 
% &\geq 1 - \exp\left(-\frac{d^{1-\frac{C^2}{2}}}{\sqrt{\frac{2\pi}{e}}d^\frac{C^2}{2}+2}\right)\\
% &\geq 1 - \exp\left(-\frac{2^{1-\frac{C^2}{2}}}{\sqrt{\frac{2\pi}{e}}2^\frac{C^2}{2}+2}\right)\\
% &\geq 0.25278
% \end{align}
% Putting all together, we have the stated bound.
% \end{proof}

\begin{theorem}
Let $d\geq2$ and $X_i$ independent Gaussian variables $N(0,\sigma^2)$. Then
\begin{align*}
E\left[\max_{i=1,\cdots,d} X_i\right] 
&\geq \sigma \left(1 - \exp\left(-\frac{\sqrt{\log d}}{2.6 \sqrt{2\pi}}\right)\right) \left(\sqrt{2 \log d-2\log \log d} +\sqrt\frac{2}{\pi}\right) -\sqrt{\frac{2}{\pi}} \sigma \\
&\geq 0.19 \sigma \sqrt{\log d} - 0.68 \sigma.
\end{align*}
\end{theorem}
\begin{proof}

Define the event $A$ equal to the case that at least one of the $X_i$ is greater or equal than $C \sigma \sqrt{\log d}$, where $C=\sqrt{2-\frac{\log \log d}{\log d}}$. With this choice we have that $d^{1-\frac{C^2}{2}}=\log d$. Also, the function $f(d)=\sqrt{2-\frac{2\log \log d}{\log d}}$ has a minimum in $d=e^e$, hence $1.27 \leq f(15) \leq C\leq f(2)\leq 1.6$.
%
\begin{align*}
E[\max_{i=1,\cdots,d} X_i] 
&= E[\max_{i=1,\cdots,d} X_i| A ] P(A) + E[\max_{i=1,\cdots,d} X_i| not A ] (1-P(A)) \\
&\geq E[\max_{i=1,\cdots,d} X_i| A ] P(A) + E[\max_{i=1,\cdots,d} X_i| not A ] (1-P(A))\\
&\geq E[\max_{i=1,\cdots,d} X_i| A ] P(A) + E[X_1| not A ](1-P(A)) \\
&= E[\max_{i=1,\cdots,d} X_i| A ] P(A) + E[X_1| X_1\leq C \sigma \sqrt{\log d}] (1-P(A))\\
&\geq E[\max_{i=1,\cdots,d} X_i| A ] P(A) + E[X_1| X_1\leq 0](1-P(A)) \\
&\geq E[\max_{i=1,\cdots,d} X_i| A ] P(A) - \sqrt{\frac{2}{\pi}} \sigma (1-P(A))\\
&\geq C \sigma \sqrt{\log d} P(A) - \sqrt{\frac{2}{\pi}} \sigma (1-P(A))\\
&= \sigma \left(C\sqrt{\log d} +\sqrt{\frac{2}{\pi}}\right)P(A) -  \sqrt{\frac{2}{\pi}} \sigma.
\end{align*}

Now let's focus on $P(A)$. It is equal to
\begin{align*}
P(A) 
&= 1-P(X_1\leq C \sigma \sqrt{\log d})^d \\
&= 1-(1-P(X_1\geq C \sigma \sqrt{\log d}))^d \\
&\geq 1-\left(1-\exp\left(-\frac{C^2 \log d}{2}\right) \frac{1}{\sqrt{2\pi}C \sqrt{\log d}+2}\right)^d \\
&= 1-\left(1-\frac{d^{-\frac{C^2}{2}}}{\sqrt{2\pi}C \sqrt{\log d}+2}\right)^d \\
&= 1- \exp\left(d \log\left(1-\frac{d^{-\frac{C^2}{2}}}{\sqrt{2\pi}C \sqrt{\log d}+2}\right)\right) \\
&\geq 1- \exp\left(d \log\left(1-\frac{d^{-\frac{C^2}{2}}}{1.6 \sqrt{2\pi} \sqrt{\log d}+2}\right)\right) \\
&\geq 1 - \exp\left(-\frac{d^{1-\frac{C^2}{2}}}{1.6 \sqrt{2\pi} \sqrt{\log d}+2}\right).
\end{align*}
where in the last inequality we used the elementary inequality $\log(1-\frac{1}{x}) \leq -\frac{1}{x}, \forall  0\leq x>1$.
We now use the fact that $C=\sqrt{2- \frac{2 \log \log d}{\log d}}$, to have
\begin{align*}
P(A) 
& \geq 1 - \exp\left(-\frac{d^{1-\frac{C^2}{2}}}{1.6 \sqrt{2\pi} \sqrt{\log d}+2}\right) \\
& = 1 - \exp\left(-\frac{\log d}{1.6 \sqrt{2\pi} \sqrt{\log d}+2}\right) \\
& \geq 1 - \exp\left(-\frac{\sqrt{\log d}}{2.6 \sqrt{2\pi}}\right),
\end{align*}
where in the last equality we used the fact that $\sqrt{2\pi} \sqrt{\log d} > 2$ for $d\geq 2$.
 
Putting all together, we have the first stated bound.
\end{proof}

\section{Binomial Case}

In the Binomial case, we expect the discrete nature of the variable to play a role. In fact, we expect it to behave like a Gaussian only when the number of draws goes to infinity.
We will quantify this intuition with the following function.
Define the function $\psi:[-\frac{1}{2},\frac{1}{2}] \rightarrow \R$ as
\[
\psi(x):=\frac{D(\frac{1}{2}+x||\frac{1}{2})}{2 x^2}~.
\]
The function $\psi(x)$ satisfies the following properties
\begin{itemize}
\item $\psi(x)=\psi(-x)$
\item $\psi(x)$ is increasing for $0\leq x \leq\frac{1}{2}$.
\item $\psi(0)=1$
\item $\psi(0.5)=2 \log(2) \approx 1.3863$
\end{itemize}

As in the Gaussian case, for the Binomial case we need lower bound to the probability tail.
We will use the next Theorem from \cite{nOrabona13}, whose proof is the Appendix for completeness.

\begin{theorem}
\label{lemma:bin}
Let $n\geq2$ an even number and $Z$ a Binomial random variable $B(n,\frac{1}{2})$. Then for any $k \in \Nat_0$ such that $k\leq \frac{1}{2}n-1$, we have
\[
P\left( Z \geq \frac{1}{2} n + k\right) 
\geq  \frac{\exp\left(-n D(\frac{1}{2}+\frac{k}{n}||\frac{1}{2})\right)}{2 \exp\left(\frac{1}{6}\right)} \frac{\sqrt{2 \pi}}{(\pi-1)y+\sqrt{y^2+2 \pi}},
\]
where $D(p||q)=p \log \frac{p}{q}+(1-p) \log\frac{1-p}{1-q}$ and $y=\frac{2 k}{\sqrt{n}}$.
\end{theorem}
%
\begin{cor}
Let $Z \sim B(n, 1/2)$ and $k \geq 1$ and $n\geq2$ even. Then
\[
P(Z \geq \frac{1}{2} n + k-1) \geq \exp\left(-\frac{1}{6}\right) \exp\left(- 2 \psi\left(\frac{k}{n}\right) \frac{k^2}{n} \right) \frac{1}{\sqrt{2\pi} \frac{2 k}{\sqrt{n}} + 2 }~.
\]
\end{cor}
\begin{proof}
\begin{align*}
P(&Z \geq  \frac{1}{2} n + k-1) \\
& = P(Z\geq \frac{1}{2} n + \lceil k -1\rceil) \\
& \geq \frac{\exp\left(-n D(\frac{1}{2}+\frac{\lceil k -1\rceil}{n}||\frac{1}{2})\right)}{2 \exp\left(\frac{1}{6}\right)} \frac{\sqrt{2 \pi}}{(\pi-1)\frac{2\lceil k -1\rceil}{\sqrt{n}}+\sqrt{\left(\frac{2\lceil k -1\rceil}{\sqrt{n}}\right)^2+2 \pi}} \\
& \geq \frac{\exp\left(-n D(\frac{1}{2}+\frac{k}{n}||\frac{1}{2})\right)}{2 \exp\left(\frac{1}{6}\right)} \frac{\sqrt{2 \pi}}{(\pi-1)\frac{2k}{\sqrt{n}}+\sqrt{\left(\frac{2k}{\sqrt{n}}\right)^2+2 \pi}} \\
& = \frac{\exp\left(- 2 \psi(\frac{k}{n}) \frac{k^2}{n} \right)}{2 \exp\left(\frac{1}{6}\right)} \frac{\sqrt{2 \pi}}{(\pi-1)\frac{2k}{\sqrt{n}}+\sqrt{\left(\frac{2k}{\sqrt{n}}\right)^2+2 \pi}} \\
& \geq \frac{\exp\left(- 2 \psi(\frac{k}{n}) \frac{k^2}{n} \right)}{2 \exp\left(\frac{1}{6}\right)} \frac{\sqrt{2 \pi}}{(\pi-1)\frac{2k}{\sqrt{n}}+\sqrt{\left(\frac{2k}{\sqrt{n}}\right)^2+2 \pi}} \\
& \geq \exp\left(-\frac{1}{6}\right) \exp\left(- 2 \psi\left(\frac{k}{n}\right) \frac{k^2}{n} \right) \frac{1}{\sqrt{2\pi} \frac{2 k}{\sqrt{n}} + 2 },
\end{align*}
where in the second equality we used Theorem~\ref{lemma:bin}.
\end{proof}

\begin{theorem}
\label{theo:max_bin}
Let $2 \leq d \leq \exp(\frac{n}{4})$, $n\geq 10$ even and $Z_i$ independent Binomial variables $B(n,1/2)$. Then
\begin{align*}
E\left[ \max_{i=1,\cdots,d} (2 Z_i - n)\right]
&\geq \frac{1}{\sqrt{\psi\left(\frac{1.6\sqrt{\log d}}{2 \sqrt{n}}\right)}}\sqrt{n}\left(1 - \exp\left(-\frac{\sqrt{\log d}}{3.1 \sqrt{2\pi}}\right)\right) \left(\sqrt{2 \log d -\log \log d}-1\right) -\sqrt{n} \\
&\geq 0.13 \sqrt{n \log d} - 2 \sqrt{n}.
\end{align*}
\end{theorem}
%
\begin{proof}
Define $X_i= 2 Z_i-n$.
Define the event $A$ equal to the case that at least one of the $X_i$ is greater or equal than $C \sqrt{n \log d}-2$, where $C=\frac{1}{\sqrt{\psi\left(\frac{\sqrt{\log d}}{2 \sqrt{n}}\right)}}\sqrt{2-\frac{\log \log d}{\log d}}$. Also, the function $f(d)=\sqrt{2-\frac{\log \log d}{\log d}}$ has a minimum in $d=e^e$, hence $1.08 \leq \frac{f(15)}{\sqrt{2 \log 2}} \leq C\leq f(2)\leq 1.6$.

Notice that the condition on $n$ and $d$ assures that $\frac{C \sqrt{n \log d}}{2}>1$ and $\frac{C \sqrt{n \log d}}{2}\leq \frac{1}{2} n -1$.
%
\begin{align*}
E[\max_{i=1,\cdots,d} X_i] 
&= E[\max_{i=1,\cdots,d} X_i| A ] P(A) + E[\max_{i=1,\cdots,d} X_i| not A ] (1-P(A)) \\
&\geq E[\max_{i=1,\cdots,d} X_i| A ] P(A) + E[\max_{i=1,\cdots,d} X_i| not A ] (1-P(A))\\
&\geq E[\max_{i=1,\cdots,d} X_i| A ] P(A) + E[X_1| not A ](1-P(A)) \\
&= E[\max_{i=1,\cdots,d} X_i| A ] P(A) + E[X_1| X_1\leq C \sigma \sqrt{\log d}] (1-P(A))\\
&\geq E[\max_{i=1,\cdots,d} X_i| A ] P(A) + E[X_1| X_1\leq 0](1-P(A)) \\
&\geq (C \sigma \sqrt{n \log d} -2) P(A) + E[X_1| X_1\leq 0] (1-P(A)).
\end{align*}

First, we lower bound $E[X_1| X_1\leq 0]$. Using the fact that $X_1$ is symmetric and with zero mean, we have
\begin{align*}
E[X_1| X_1\leq 0] &= \sum_{k=-n}^0 k P(k | k\leq 0) = 2 \sum_{k=-n}^0 k P(k) \\
&= - \sum_{k=-n}^n |k| P(k) = - E[|X_1|] \geq - \sqrt{Var[X_1]}\\
&= -\sqrt{n}.
\end{align*}

Now let's focus on $P(A)$. It is equal to
\begin{align*}
P(A) 
&= 1-P(X_1 < C \sqrt{n \log d}-2)^d \\
&= 1-(1-P(X_1\geq C \sqrt{n \log d}-2))^d \\
&= 1-(1-P(Z_1\geq \frac{C \sqrt{n \log d}}{2} +\frac{n}{2}-1))^d \\
&\geq 1-\left(1-\exp\left(- \psi\left(\frac{C \sqrt{\log d}}{2 \sqrt{n}}\right) \frac{C^2 \log d}{2}\right) \frac{\exp\left(-\frac{1}{6}\right)}{\sqrt{2\pi}C \sqrt{\log d}+2}\right)^d \\
&\geq 1-\left(1-\exp\left(- \psi\left(\frac{1.6 \sqrt{\log d}}{2 \sqrt{n}}\right) \frac{C^2 \log d}{2}\right) \frac{\exp\left(-\frac{1}{6}\right)}{\sqrt{2\pi}C \sqrt{\log d}+2}\right)^d \\
&= 1-\left(1-\frac{\exp\left(-\frac{1}{6}\right) d^{-\frac{C^2}{2} \psi\left(\frac{1.6 \sqrt{\log d}}{2 \sqrt{n}}\right)}}{\sqrt{2\pi}C \sqrt{\log d}+2}\right)^d \\
&= 1- \exp\left(d \log\left(1-\frac{\exp\left(-\frac{1}{6}\right) d^{-\frac{C^2}{2} \psi\left(\frac{1.6 \sqrt{\log d}}{2 \sqrt{n}}\right)}}{\sqrt{2\pi}C \sqrt{\log d}+2}\right)\right) \\
&\geq 1- \exp\left(d \log\left(1-\frac{\exp\left(-\frac{1}{6}\right) d^{-\frac{C^2}{2} \psi\left(\frac{1.6 \sqrt{\log d}}{2 \sqrt{n}}\right)}}{1.6 \sqrt{2\pi} \sqrt{\log d}+2}\right)\right) \\
&\geq 1 - \exp\left(-\frac{\exp\left(-\frac{1}{6}\right) d^{1-\frac{C^2}{2} \psi\left(\frac{1.6 \sqrt{\log d}}{2 \sqrt{n}}\right)}}{1.6 \sqrt{2\pi} \sqrt{\log d}+2}\right).
\end{align*}
where in the last inequality we used the elementary inequality $\log(1-\frac{1}{x}) \leq -\frac{1}{x}, \forall  0\leq x>1$.
We now use the fact that $C=\frac{1}{\sqrt{\psi\left(\frac{1.6 \sqrt{\log d}}{2 \sqrt{n}}\right)}}\sqrt{2- \frac{\log \log d}{\log d}}$ that implies $d^{1-\frac{C^2}{2} \psi\left(\frac{1.6 \sqrt{\log d}}{2 \sqrt{n}}\right)}=\log d$. Hence, we obtain
\begin{align*}
P(A) 
& \geq 1 - \exp\left(-\frac{\exp\left(-\frac{1}{6}\right) d^{1-\frac{C^2}{2} \psi\left(\frac{1.6 \sqrt{\log d}}{2 \sqrt{n}}\right)}}{1.6 \sqrt{2\pi} \sqrt{\log d}+2}\right) \\
& = 1 - \exp\left(-\frac{\exp\left(-\frac{1}{6}\right) \log d}{1.6 \sqrt{2\pi} \sqrt{\log d}+2}\right) \\
& \geq 1 - \exp\left(-\frac{\exp\left(-\frac{1}{6}\right) \sqrt{\log d}}{2.6 \sqrt{2\pi}}\right)\\
& \geq 1 - \exp\left(-\frac{\sqrt{\log d}}{3.1 \sqrt{2\pi}}\right),
\end{align*}
where in the last equality we used the fact that $\sqrt{2\pi} \sqrt{\log d} > 2$ for $d\geq 2$.

Putting all together, we have the stated bound.
\end{proof}

\section{Lower Bound for Regret in the Experts Setting}

We present here our main result.

\begin{theorem}
Let the outcome space $\mathcal{Y}=\{0,1\}$, and the decision space $\mathcal{D}=[0,1]$, and $\ell$ the absolute loss, $\ell(p,q)=|p-q|$. Let $n\geq4$ and even, and $2\leq d \leq \exp(\frac{n}{4})$.
Define
\[
f(n,d)=\frac{1}{2}\frac{1}{\sqrt{\psi\left(\frac{1.6 \sqrt{\log d}}{2 \sqrt{n}}\right)}}\sqrt{n}\left(1 - \exp\left(-\frac{\sqrt{\log d}}{3.1 \sqrt{2\pi}}\right)\right) \left(\sqrt{2 \log d -\log \log d}-1\right) -\frac{1}{2}\sqrt{n}~.
\]
Then
\[
Regret^{(d)}(n)\geq f(n,d)
\]
and
\[
\sup_{n,d} \frac{f(n,d)}{\sqrt{\frac{n}{2} \log d}} \geq 1~.
\]
\end{theorem}
%
\begin{proof}
Proceeding as in the proof of Theorem~3.7 in~\citep{Cesa-BianchiL06} we only need to show that
\[
\frac{1}{2} E\left[ \max_{i=1,\cdots,d} Z_i\right] \geq f(n,d),
\]
where $Z_i= 2 X_i - n$ and $X_i \sim B(n, \frac{1}{2})$. We simply do it through Theorem~\ref{theo:max_bin}.
For the second statement, we use the fact that $\lim_{n\rightarrow\infty} \psi\left(\frac{1.6 \sqrt{\log d}}{2 \sqrt{n}}\right) = 1$.
\end{proof}

The theorem proves a non-asymptotic lower bounds, while at the same time recovering the optimal constant of the asymptotic one in \citet{Cesa-BianchiL06}. Also, differently from the asymptotic lower bound in \citet{Cesa-BianchiL06}, we do not need to take the limit for $n$ that goes to infinity.

\bibliographystyle{plainnat}
\bibliography{learning}

\appendix

\section{Proof of Theorem~\ref{theo:max_bin}}
\begin{proof}
We use Theorem~2 in \cite{McKay1989}, that specialized to our case says that
\begin{equation}
\label{eq:bin_1}
P\left( Z \geq  \frac{1}{2} n + k  \right) 
\geq \sqrt{n} \binom{n-1}{ \frac{1}{2} n + k -1} 2^{-n} \frac{Q(y)}{\phi(y)},
\end{equation}
where $\phi(x)$ is the unit variance, zero mean Gaussian, $\frac{1}{\sqrt{2 \pi}} \exp(-\frac{x^2}{2})$ and $Q(x)$ is its CDF, $\int_{x}^{+\infty} \phi(u) du$.

We lower bounding the ratio $\frac{Q(y)}{\phi(y)}$ using Lemma~\ref{lemma:boyd}:
\[
\frac{Q(y)}{\phi(y)} 
= \exp\left(\frac{x^2}{2}\right) \int_{x}^{+\infty} \exp\left(-\frac{t^2}{2}\right) dt
\geq \frac{\pi}{(\pi-1)x+\sqrt{x^2+2 \pi}}.
\]

To bound the binomial coefficient we make use of the following Stirling approximation, for any $n\geq 1$,
\[
\sqrt{2 \pi n} n^n \exp(-n) < n! < \exp\left(\frac{1}{12}\right)\sqrt{2 \pi n} n^n \exp(-n)~.
\]
Hence, for any $n \geq 2$ and $1\leq q \leq n-1$, after some algebra we obtain
\begin{align*}
{n \choose q} 
&\geq \frac{1}{\exp\left(\frac{1}{6}\right) \sqrt{2 \pi}} \left(\frac{n}{n-q}\right)^{n-q} \left(\frac{n}{q}\right)^{q} \sqrt{\frac{n}{q(n-q)}} \\
&\geq \frac{1}{\exp\left(\frac{1}{6}\right) \sqrt{2 \pi}} 2^n \exp\left(-n D\left(\frac{q}{n}||\frac{1}{2}\right)\right) \sqrt{\frac{n}{q(n-q)}}.
\end{align*}
where in the equality we used the definition of $D$.
Also, we have
\begin{equation}
\label{eq:bin_3}
{n-1 \choose \frac{1}{2} n + k - 1} = {n \choose \frac{1}{2} n + k} \left(\frac{1}{2} + \frac{k}{n}\right) .
\end{equation}
Putting together \eqref{eq:bin_1}-\eqref{eq:bin_3}, and using the definition of $y$ we have
\begin{align*}
&P\left( Z \geq \frac{1}{2} n + k \right) 
\geq \frac{1}{\exp\left(\frac{1}{6}\right) \sqrt{2 \pi}} \exp\left(-n D\left(\frac{1}{2}+\frac{k}{n}||\frac{1}{2}\right)\right) \sqrt{\frac{\frac{1}{2} + \frac{k}{n}}{\frac{1}{2}-\frac{k}{n}}}  \frac{Q(y)}{\phi(y)} \\
&\geq \frac{1}{\exp\left(\frac{1}{6}\right) \sqrt{2 \pi}} \exp\left(-n D\left(\frac{1}{2}+\frac{k}{n}||\frac{1}{2}\right)\right) \frac{\pi}{(\pi-1)y+\sqrt{y^2+2 \pi}}. \qedhere
\end{align*}
\end{proof}

\end{document}