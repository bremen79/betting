\documentclass{article}

\usepackage{algorithm, algorithmic}
\usepackage{amsmath, amsthm, amssymb}
\usepackage{fullpage}
\usepackage{natbib}
\usepackage{times}

\DeclareMathOperator*{\Exp}{\mathbf{E}}
\DeclareMathOperator*{\argmin}{arg\,min}
\DeclareMathOperator*{\argmax}{arg\,max}
\DeclareMathOperator{\Regret}{Regret}

\newcommand{\field}[1]{\mathbb{#1}}
\newcommand{\R}{\field{R}}
\newcommand{\Nat}{\field{N}}
\newcommand{\Var}{\mathrm{Var}}

\newtheorem{theorem}{Theorem}
\newtheorem{lemma}[theorem]{Lemma}
\newtheorem{corrollary}[theorem]{Corollary}
\newtheorem{remark}[theorem]{Remark}
\newtheorem{proposition}[theorem]{Proposition}

\newcommand{\reals}{\mathbb{R}}
\newcommand{\sign}{{\rm sign}}

\begin{document}

\title{Optimal Non-Asymptotic Lower Bound \\ on the \\ Minimax Regret for Learning with Expert Advice}
\author{
\begin{tabular}{c@{\hskip 1in}c}
  Francesco Orabona & David Pal \\
  francesco@orabona.com & dpal@yahoo-inc.com \\
\end{tabular}
\\\\
Yahoo Labs \\
New York, NY, USA
}


\maketitle

\begin{abstract}
We prove non-asymptotic lower bounds on the expectation of the maximum of $d$
independent Gaussian variables and expectation of the maximum of $d$
independent symmetric random walks. Both lower bound recovers
the optimal leading constant in the limit.

A simple application of the lower bound for random walks is a non-asymptotic lower
bound on the minimax regret of online learning with expert advice.
\end{abstract}

\section{Introduction}

Let $X_1, X_2, \dots, X_d$ be i.i.d. Gaussian random variables $N(0,\sigma^2)$.
It easy to prove that
\begin{equation}
\label{equation:upper-bound-on-maximum-of-gaussians}
\Exp \left[ \max_{1 \le i \le d} X_i \right] \le \sigma \sqrt{2 \ln d} \qquad \text{for any $d \ge 1$} \; .
\end{equation}
It is also well known that
\begin{equation}
\label{equation:limit-maximum-of-gaussians}
\lim_{d \to \infty} \frac{\Exp \left[ \max_{1 \le i \le d} X_i \right]}{\sigma \sqrt{2 \ln d}} = 1 \; .
\end{equation}
In section~\ref{section:maximum-of-gaussians}, we prove a non-asymptotic
$\Omega(\sigma \sqrt{\log d})$ lower bound on $\Exp[\max_{1 \le i \le d} X_i]$. The leading term
of the lower bound is asymptotically $\sqrt{2 \ln d}$. In other words, the lower bound implies~\eqref{equation:limit-maximum-of-gaussians}.

Discrete analog of a Gaussian random variable is the symmetric random walk. Recall that a random walk $Z^{(n)}$
of length $n$ is a sum $Z^{(n)} = Y_1 + Y_2 + \dots + Y_n$ of $n$ i.i.d. Rademacher variables, which have probability distribution
$\Pr[Y_i = +1] = \Pr[Y_i = -1] = 1/2$. We consider $d$ independent symmetric random walks $Z^{(n)}_1, Z^{(n)}_2, \dots, Z^{(n)}_d$ of length $n$.
Anologously to \eqref{equation:upper-bound-on-maximum-of-gaussians}, it is easy to prove that
\begin{equation}
\label{equation:upper-bound-on-maximum-of-random-walks}
\Exp \left[ \max_{1 \le i \le d} Z^{(n)}_i \right] \le \sqrt{2 n \ln d} \qquad \text{for any $n \ge 0$ and any $d \ge 1$}\; .
\end{equation}
Note that $\sigma^2$ in is replaced by $\Var(Z^{(n)}_i) = n$. By central limit theorem $\frac{Z^{(n)}_i}{\sqrt{n}}$
as $n \to \infty$ converges in distribution to $N(0,1)$. From this fact, it possible to prove
the analog of \eqref{equation:limit-maximum-of-gaussians},
\begin{equation}
\label{equation:limit-maximum-of-random-walks}
\lim_{d \to \infty} \lim_{n \to \infty} \frac{\Exp\left[ \max_{1 \le i \le d} Z^{(n)}_i \right]}{\sqrt{2 n \ln d}} = 1 \; .
\end{equation}
We prove a non-asymptotic $\Omega(\sqrt{n \log d})$ lower bound on $\Exp\left[ \max_{1 \le i \le d} Z^{(n)}_i \right]$.
Same as for the Gaussian case, the leading term of the lower bound is asymptotically $\sqrt{2 n \ln d}$
matching~\eqref{equation:limit-maximum-of-random-walks}.

In section~\ref{section:experts}, we show a simple application of the lower
bound on $\Exp\left[\max_{1 \le i \le d} Z^{(n)}_i \right]$ to the problem of learning with
expert advice.  This problem was extensively studied in the online learning
literature; see~\citep{Cesa-BianchiL06}.  Our bound is optimal in the sense
that for larg $d$ and large $n$ it recovers the right leading constant.

\section{Maximum of Gaussians}
\label{section:maximum-of-gaussians}

It is well known that the maximum of Gaussian variables converges to a Gumbel
distribution. Here we quantify the rate of convergence for the expectation of
maximum of independent Gaussian variables.

We start with a lower bound on the Mill's ratio of Gaussian variables. Recall that Mill's ratio of a random variable $X$
with density function $f(x)$ is the ratio $\frac{\Pr[X > x]}{f(x)}$ and that density function of $X \sim N(0,\sigma^2)$
is $f(x) = \frac{1}{\sigma \sqrt{2 \pi}} \exp\left( - \frac{x^2}{2 \sigma^2} \right)$.

\begin{lemma}[Mill's ratio for a Gaussian; see~\cite{Boyd-1959}]
\label{lemma:boyd}
For any $x \ge 0$,
\[
\exp\left(\frac{x^2}{2}\right) \int_x^{\infty} \exp\left(-\frac{t^2}{2}\right) dt
\ge \frac{\pi}{(\pi-1)x+\sqrt{x^2+2 \pi}}.
\]
\end{lemma}

The bound on Mill's ratio can be turned into a lower bound on the tail of the distribution.

\begin{corrollary}[Lower Bound on Gaussian Tail]
Let $X \sim N(\mu, \sigma^2)$ and $t \ge 0$. Then,
$$
\Pr[X \ge t] \ge \exp\left(-\frac{t^2}{2 \sigma^2}\right) \frac{1}{\sqrt{2\pi}\frac{t}{\sigma}+2}.
$$
\end{corrollary}

\begin{proof}
We have
\begin{align*}
\Pr[X \ge t]
& = \frac{1}{\sigma \sqrt{2 \pi}} \int_t^{\infty} \exp \left( -\frac{x^2}{2 \sigma^2} \right) d x \\
& = \frac{1}{\sqrt{2 \pi}} \int_\frac{t}{\sigma}^{\infty} \exp \left( -\frac{x^2}{2} \right) d x \\
& \ge \frac{1}{\sqrt{2 \pi}} \exp\left(-\frac{t^2}{2 \sigma^2}\right) \frac{\pi}{(\pi-1)\frac{t}{\sigma}+\sqrt{\frac{t^2}{\sigma^2}+2 \pi}} \\
& \ge \frac{1}{\sqrt{2}} \exp\left(-\frac{t^2}{2 \sigma^2}\right) \frac{\sqrt{\pi}}{\pi\frac{t}{\sigma}+\sqrt{2 \pi}} \\
& = \exp\left(-\frac{t^2}{2 \sigma^2}\right) \frac{1}{\sqrt{2\pi}\frac{t}{\sigma}+2} \; ,
\end{align*}
where in the first inequality we used Lemma~\ref{lemma:boyd}.
\end{proof}

\begin{theorem}
Let $X_1, X_2, \dots, X_d$ be independent Gaussian random variables $N(0,\sigma^2)$. For $d \ge 2$,
\begin{align}
\Exp \left[\max_{1 \le i \le d} X_i\right]
& \ge \sigma \left(1 - \exp\left(-\frac{\sqrt{\ln d}}{6.35}\right)\right) \left(\sqrt{2 \ln d - 2 \ln \ln d} +\sqrt\frac{2}{\pi}\right) -\sqrt{\frac{2}{\pi}} \sigma \label{equation:maximum-of-gaussians-lower-bound-1} \\
& \ge 0.13 \sigma \sqrt{\ln d} - 0.7 \sigma \label{equation:maximum-of-gaussians-lower-bound-2} \; .
\end{align}
\end{theorem}

\begin{proof}
Let $A$ be the event that at least one of the $X_i$ is greater than $C \sigma
\sqrt{\ln d}$ where $C = C(d) = \sqrt{2 - \frac{2 \ln \ln d}{\ln d}}$. We
denote by $\overline{A}$ the complement of this event. We have
\begin{align}
\Exp \left[ \max_{1 \le i \le d} X_i \right]
& = \Exp \left[ \max_{1 \le i \le d} X_i ~\bigg|~ A \right] \cdot \Pr[A] + \Exp \left[ \max_{1 \le i \le d} X_i ~\bigg|~ \overline{A} \right] \cdot \Pr \left[ \overline{A} \right] \notag \\
& \ge \Exp \left[ \max_{1 \le i \le d} X_i ~\bigg|~ A \right] \cdot \Pr[A] + \Exp \left[ X_1 ~\big|~ \overline{A} \right] \cdot \Pr[\overline{A}] \notag \\
& = \Exp \left[ \max_{1 \le i \le d} X_i~\bigg|~ A \right] \cdot \Pr[A] + \Exp \left[ X_1~|~ X_1 \le C \sigma \sqrt{\ln d} \right] \cdot \Pr[\overline{A}] \notag \\
& \ge \Exp \left[ \max_{1 \le i \le d} X_i~\bigg|~ A \right] \cdot \Pr[A] + \Exp[X_1~|~ X_1 \le 0] \cdot \Pr[\overline{A}] \notag \\
& \ge \Exp \left[ \max_{1 \le i \le d} X_i~\bigg|~ A \right] \cdot \Pr[A] - \sigma \sqrt{\frac{2}{\pi}} \cdot \Pr[\overline{A}] \notag \\
& \ge C \sigma \sqrt{\ln d} \cdot \Pr[A] - \sigma \sqrt{\frac{2}{\pi}} (1 - \Pr[A]) \notag \\
& = \sigma \left(C\sqrt{\ln d} + \sqrt{\frac{2}{\pi}}\right) \Pr[A] -  \sigma \sqrt{\frac{2}{\pi}} \label{equation:maximum-of-gaussians-lower-bound-3}
\end{align}
where we used that $\Exp[X_1 ~|~ X_1 \le 0] = 2 \int_{-\infty}^0 x f(x) dx = 2 \int_{-\infty}^0 \frac{x}{\sigma \sqrt{2\pi}} \exp \left(- \frac{x^2}{2\sigma^2} \right) = - \sigma \sqrt{\frac{2}{\pi}}$.

It remains to lower bound $\Pr[A]$, which we do as follows
\begin{align}
\Pr[A]
& = 1 - \Pr[\overline{A}] \notag \\
& = 1 - \prod_{i=1}^d \Pr \left[ X_i \le C \sigma \sqrt{\ln d} \right]  \notag \\
& = 1 - \left(1 - \Pr\left[ X_1 > C \sigma \sqrt{\ln d} \right] \right)^d \notag \\
& \ge 1 - \exp\left(-d \cdot \Pr\left[X_1 \ge C \sigma \sqrt{\ln d} \right]\right) \notag \\
& \ge 1 - \exp\left(-d \exp\left(-\frac{C^2 \ln d}{2}\right) \frac{1}{\sqrt{2\pi}C \sqrt{\ln d}+2} \right) \notag \\
& = 1 - \exp\left(-\frac{d^{1-\frac{C^2}{2}}}{C \sqrt{2\pi \ln d}+2}\right) \label{equation:maximum-of-gaussians-lower-bound-4} \; .
\end{align}
where in the first inequality we used the elementary inequality $1 - x \le \exp(-x)$ valid for all $x \in \R$.

Since $C = \sqrt{2 - \frac{2 \ln \ln d}{\ln d}}$ we have $d^{1-\frac{C^2}{2}} = \ln d$. Subtistuting this into \eqref{equation:maximum-of-gaussians-lower-bound-4}, we get
\begin{equation}
\label{equation-maximum-of-gaussians-lower-bound-5}
\Pr[A] \ge 1 - \exp\left(-\frac{\ln d}{C \sqrt{2\pi \ln d}+2}\right) = 1 - \exp\left(-\frac{\sqrt{\ln d}}{C \sqrt{2\pi}+2}\right) \; .
\end{equation}
The function $C(d)$ is decreasing on the interval $[1,e^e]$, increasing on $[e^e, \infty)$, and $\lim_{d \to \infty} C(d) = \sqrt{2}$. From these properties
we can deduce that $C(d) \le \max\{C(2), \sqrt{2}\} \le 1.75$ for any $d \in [2,\infty)$. Therefore, $C\sqrt{2 \pi} + 2 \le 6.35$ and hence
\begin{equation}
\label{equation:maximum-of-gaussians-lower-bound-6}
\Pr[A] \ge 1 - \exp\left(-\frac{\sqrt{\ln d}}{6.35}\right) \; .
\end{equation}
Inequalities \eqref{equation:maximum-of-gaussians-lower-bound-3} and \eqref{equation:maximum-of-gaussians-lower-bound-6} together imply bound \eqref{equation:maximum-of-gaussians-lower-bound-1}.

Bound \eqref{equation:maximum-of-gaussians-lower-bound-2} is obtained from \eqref{equation:maximum-of-gaussians-lower-bound-1} by noticing that
\begin{align*}
& \sigma \left(1 - \exp\left(-\frac{\sqrt{\ln d}}{6.35}\right)\right) \left(\sqrt{2 \ln d - 2 \ln \ln d} +\sqrt\frac{2}{\pi}\right) -\sqrt{\frac{2}{\pi}} \sigma \\
& = \sigma \left(1 - \exp\left(-\frac{\sqrt{\ln d}}{6.35}\right)\right) \sqrt{2 \ln d - 2 \ln \ln d} - \exp\left(-\frac{\sqrt{\ln d}}{6.35}\right) \sqrt{\frac{2}{\pi}} \sigma \\
& \ge 0.1227 \cdot \sigma \sqrt{2 \ln d - 2 \ln \ln d} - 0.7 \sigma \\
& = 0.1227 \cdot \sigma \sqrt{\ln d} \cdot C(d) - 0.7 \sigma
\end{align*}
where we used that $\exp\left(-\frac{\sqrt{\ln d}}{6.35}\right) \le 0.8773$ for any $d \ge 2$.
Since $C(d)$ has minimum at $d = e^e$, it follows that $C(d) \ge C(e^e) = \sqrt{2 - \frac{2}{e}} \ge 1.1243$ for any $d \ge 2$.
\end{proof}

\section{Binomial Case}
\label{section:maximum-of-random-walks}

In the Binomial case, we expect the discrete nature of the variable to play a role. In fact, we expect it to behave like a Gaussian only when the number of draws goes to infinity.
We will quantify this intuition with the following function.
Define the function $\psi:[-\frac{1}{2},\frac{1}{2}] \to \R$ as
$$
\psi(x) = \frac{D(\frac{1}{2}+x\|\frac{1}{2})}{2 x^2}~.
$$
The function $\psi(x)$ satisfies the following properties
\begin{itemize}
\item $\psi(x) = \psi(-x)$
\item $\psi(x)$ is increasing for $0\le x \le \frac{1}{2}$.
\item $\psi(0) = 1$
\item $\psi(0.5) = 2 \log(2) \approx 1.3863$
\end{itemize}

As in the Gaussian case, for the Binomial case we need lower bound to the probability tail.
We will use the next Theorem from \cite{nOrabona13}, whose proof is the Appendix for completeness.

\begin{theorem}
\label{lemma:bin}
Let $n \ge 2$ an even number and $Z$ a Binomial random variable $B(n,\frac{1}{2})$. Then for any $k \in \Nat_0$ such that $k\le \frac{1}{2}n-1$, we have
\[
P\left( Z \ge \frac{1}{2} n + k\right)
\ge  \frac{\exp\left(-n D(\frac{1}{2}+\frac{k}{n}\|\frac{1}{2})\right)}{2 \exp\left(\frac{1}{6}\right)} \frac{\sqrt{2 \pi}}{(\pi-1)y+\sqrt{y^2+2 \pi}},
\]
where $D(p\|q)=p \log \frac{p}{q}+(1-p) \log\frac{1-p}{1-q}$ and $y=\frac{2 k}{\sqrt{n}}$.
\end{theorem}
%
\begin{corrollary}
Let $Z \sim B(n, 1/2)$ and $k \ge 1$ and $n \ge 2$ even. Then
\[
\Pr(Z \ge \frac{1}{2} n + k-1) \ge \exp\left(-\frac{1}{6}\right) \exp\left(- 2 \psi\left(\frac{k}{n}\right) \frac{k^2}{n} \right) \frac{1}{\sqrt{2\pi} \frac{2 k}{\sqrt{n}} + 2 }~.
\]
\end{corrollary}
\begin{proof}
\begin{align*}
\Pr(Z \ge  \frac{1}{2} n + k-1)
& = \Pr(Z \ge \frac{1}{2} n + \lceil k -1\rceil) \\
& \ge \frac{\exp\left(-n D(\frac{1}{2}+\frac{\lceil k -1\rceil}{n} \| \frac{1}{2})\right)}{2 \exp\left(\frac{1}{6}\right)} \frac{\sqrt{2 \pi}}{(\pi-1)\frac{2\lceil k -1\rceil}{\sqrt{n}}+\sqrt{\left(\frac{2\lceil k -1\rceil}{\sqrt{n}}\right)^2+2 \pi}} \\
& \ge \frac{\exp\left(-n D(\frac{1}{2}+\frac{k}{n} \| \frac{1}{2})\right)}{2 \exp\left(\frac{1}{6}\right)} \frac{\sqrt{2 \pi}}{(\pi-1)\frac{2k}{\sqrt{n}}+\sqrt{\left(\frac{2k}{\sqrt{n}}\right)^2+2 \pi}} \\
& = \frac{\exp\left(- 2 \psi(\frac{k}{n}) \frac{k^2}{n} \right)}{2 \exp\left(\frac{1}{6}\right)} \frac{\sqrt{2 \pi}}{(\pi-1)\frac{2k}{\sqrt{n}}+\sqrt{\left(\frac{2k}{\sqrt{n}}\right)^2+2 \pi}} \\
& \ge \frac{\exp\left(- 2 \psi(\frac{k}{n}) \frac{k^2}{n} \right)}{2 \exp\left(\frac{1}{6}\right)} \frac{\sqrt{2 \pi}}{(\pi-1)\frac{2k}{\sqrt{n}}+\sqrt{\left(\frac{2k}{\sqrt{n}}\right)^2+2 \pi}} \\
& \ge \exp\left(-\frac{1}{6}\right) \exp\left(- 2 \psi\left(\frac{k}{n}\right) \frac{k^2}{n} \right) \frac{1}{\sqrt{2\pi} \frac{2 k}{\sqrt{n}} + 2} \; ,
\end{align*}
where in the second equality we used Theorem~\ref{lemma:bin}.
\end{proof}

\begin{theorem}
\label{theorem:max_bin}
Let $2 \le d \le \exp(\frac{n}{4})$, $n\ge 10$ even and $Z_i$ independent Binomial variables $B(n,1/2)$. Then
\begin{align*}
\Exp \left[ \max_{1 \le i \le d} (2 Z_i - n)\right]
& \ge \frac{1}{\sqrt{\psi\left(\frac{1.6\sqrt{\log d}}{2 \sqrt{n}}\right)}}\sqrt{n}\left(1 - \exp\left(-\frac{\sqrt{\log d}}{3.1 \sqrt{2\pi}}\right)\right) \left(\sqrt{2 \log d -\log \log d}-1\right) -\sqrt{n} \\
& \ge 0.13 \sqrt{n \log d} - 2 \sqrt{n}.
\end{align*}
\end{theorem}
%
\begin{proof}
Define $X_i= 2 Z_i-n$.
Define the event $A$ equal to the case that at least one of the $X_i$ is greater or equal than $C \sqrt{n \log d}-2$, where $C=\frac{1}{\sqrt{\psi\left(\frac{\sqrt{\log d}}{2 \sqrt{n}}\right)}}\sqrt{2-\frac{\log \log d}{\log d}}$. Also, the function $f(d)=\sqrt{2-\frac{\log \log d}{\log d}}$ has a minimum in $d=e^e$, hence $1.08 \le \frac{f(15)}{\sqrt{2 \log 2}} \le C\le f(2)\le 1.6$.

Notice that the condition on $n$ and $d$ assures that $\frac{C \sqrt{n \log d}}{2}>1$ and $\frac{C \sqrt{n \log d}}{2}\le \frac{1}{2} n -1$.
%
\begin{align*}
\Exp [\max_{1 \le i \le d} X_i]
& = \Exp [\max_{1 \le i \le d} X_i| A ] \Pr(A) + \Exp [\max_{1 \le i \le d} X_i| \overline{A} ] \Pr(\overline{A}) \\
& \ge \Exp[\max_{1 \le i \le d} X_i| A ] \Pr(A) + \Exp[\max_{1 \le i \le d} X_i| \overline{A} ] \Pr(\overline{A})\\
& \ge \Exp[\max_{1 \le i \le d} X_i| A ] \Pr(A) + \Exp[X_1| \overline{A} ] \Pr(\overline{A} ) \\
& = \Exp[\max_{1 \le i \le d} X_i| A ] \Pr(A) + \Exp[X_1| X_1\le C \sigma \sqrt{\log d}] \Pr(\overline{A})\\
& \ge \Exp[\max_{1 \le i \le d} X_i| A ] \Pr(A) + \Exp[X_1| X_1\le 0](1-\Pr(A)) \\
& \ge (C \sigma \sqrt{n \log d} -2) \Pr(A) + \Exp[X_1| X_1\le 0] (1-\Pr(A)).
\end{align*}

First, we lower bound $\Exp[X_1~|~ X_1 \le 0]$. Using the fact that $X_1$ is symmetric and with zero mean, we have
\begin{align*}
\Exp[X_1| X_1\le 0] &= \sum_{k=-n}^0 k \Pr(k | k\le 0) = 2 \sum_{k=-n}^0 k \Pr(k) \\
& = - \sum_{k=-n}^n |k| \Pr(k) = - \Exp[|X_1|] \ge - \sqrt{\Var[X_1]}\\
& = -\sqrt{n}.
\end{align*}

Now let's focus on $\Pr[A]$. As in the Gaussian case, we can lower bound it as
\begin{align*}
\Pr[A]
& = 1 - \Pr(X_1 < C \sqrt{n \log d}-2)^d \\
& = 1 - (1-\Pr(X_1\ge C \sqrt{n \log d}-2))^d \\
& = 1 - (1-\Pr(Z_1\ge \frac{C \sqrt{n \log d}}{2} +\frac{n}{2}-1))^d \\
& \ge 1-\exp\left(-d \Pr(Z_1\ge \frac{C \sqrt{n \log d}}{2} +\frac{n}{2}-1) \right) \\
& \ge 1 - \exp\left(-\frac{\exp\left(-\frac{1}{6}\right) d^{1-\frac{C^2}{2} \psi\left(\frac{C \sqrt{\log d}}{2 \sqrt{n}}\right)}}{C \sqrt{2\pi} \sqrt{\log d}+2}\right) \\
& \ge 1 - \exp\left(-\frac{\exp\left(-\frac{1}{6}\right) d^{1-\frac{C^2}{2} \psi\left(\frac{1.6 \sqrt{\log d}}{2 \sqrt{n}}\right)}}{1.6 \sqrt{2\pi} \sqrt{\log d}+2}\right).
%
% &\ge 1-\left(1-\exp\left(- \psi\left(\frac{C \sqrt{\log d}}{2 \sqrt{n}}\right) \frac{C^2 \log d}{2}\right) \frac{\exp\left(-\frac{1}{6}\right)}{\sqrt{2\pi}C \sqrt{\log d}+2}\right)^d \\
% &\ge 1-\left(1-\exp\left(- \psi\left(\frac{1.6 \sqrt{\log d}}{2 \sqrt{n}}\right) \frac{C^2 \log d}{2}\right) \frac{\exp\left(-\frac{1}{6}\right)}{\sqrt{2\pi}C \sqrt{\log d}+2}\right)^d \\
% &= 1-\left(1-\frac{\exp\left(-\frac{1}{6}\right) d^{-\frac{C^2}{2} \psi\left(\frac{1.6 \sqrt{\log d}}{2 \sqrt{n}}\right)}}{\sqrt{2\pi}C \sqrt{\log d}+2}\right)^d \\
% &= 1- \exp\left(d \log\left(1-\frac{\exp\left(-\frac{1}{6}\right) d^{-\frac{C^2}{2} \psi\left(\frac{1.6 \sqrt{\log d}}{2 \sqrt{n}}\right)}}{\sqrt{2\pi}C \sqrt{\log d}+2}\right)\right) \\
% &\ge 1- \exp\left(d \log\left(1-\frac{\exp\left(-\frac{1}{6}\right) d^{-\frac{C^2}{2} \psi\left(\frac{1.6 \sqrt{\log d}}{2 \sqrt{n}}\right)}}{1.6 \sqrt{2\pi} \sqrt{\log d}+2}\right)\right) \\
% &\ge 1 - \exp\left(-\frac{\exp\left(-\frac{1}{6}\right) d^{1-\frac{C^2}{2} \psi\left(\frac{1.6 \sqrt{\log d}}{2 \sqrt{n}}\right)}}{1.6 \sqrt{2\pi} \sqrt{\log d}+2}\right).
\end{align*}
%where in the last inequality we used the elementary inequality $\log(1-\frac{1}{x}) \le -\frac{1}{x}, \forall  0\le x>1$.
We now use the fact that $C=\frac{1}{\sqrt{\psi\left(\frac{1.6 \sqrt{\log d}}{2 \sqrt{n}}\right)}}\sqrt{2- \frac{\log \log d}{\log d}}$ that implies $d^{1-\frac{C^2}{2} \psi\left(\frac{1.6 \sqrt{\log d}}{2 \sqrt{n}}\right)}=\log d$. Hence, we obtain
\begin{align*}
\Pr(A)
& \ge 1 - \exp\left(-\frac{\exp\left(-\frac{1}{6}\right) d^{1-\frac{C^2}{2} \psi\left(\frac{1.6 \sqrt{\log d}}{2 \sqrt{n}}\right)}}{1.6 \sqrt{2\pi} \sqrt{\log d}+2}\right) \\
& = 1 - \exp\left(-\frac{\exp\left(-\frac{1}{6}\right) \log d}{1.6 \sqrt{2\pi} \sqrt{\log d}+2}\right) \\
& \ge 1 - \exp\left(-\frac{\exp\left(-\frac{1}{6}\right) \sqrt{\log d}}{2.6 \sqrt{2\pi}}\right)\\
& \ge 1 - \exp\left(-\frac{\sqrt{\log d}}{3.1 \sqrt{2\pi}}\right),
\end{align*}
where in the last equality we used the fact that $\sqrt{2\pi} \sqrt{\log d} > 2$ for $d\ge 2$.

Putting all together, we have the stated bound.
\end{proof}

\section{Lower Bound for Regret in the Experts Setting}
\label{section:experts}

We present here our main result.

\begin{theorem}
Let the outcome space $\mathcal{Y}=\{0,1\}$, and the decision space
$\mathcal{D}=[0,1]$, and $\ell$ the absolute loss, $\ell(p,q)=|p-q|$. Let $n
\ge 4$ and even, and $2\le d \le \exp(\frac{n}{4})$.  Define
\[
f(n,d)=\frac{1}{2}\frac{1}{\sqrt{\psi\left(\frac{1.6 \sqrt{\log d}}{2 \sqrt{n}}\right)}}\sqrt{n}\left(1 - \exp\left(-\frac{\sqrt{\log d}}{3.1 \sqrt{2\pi}}\right)\right) \left(\sqrt{2 \log d -\log \log d}-1\right) -\frac{1}{2}\sqrt{n}~.
\]
Then
\[
\Regret^{(d)}(n)\ge f(n,d)
\]
and
\[
\sup_{n,d} \frac{f(n,d)}{\sqrt{\frac{n}{2} \log d}} \ge 1~.
\]
\end{theorem}
%
\begin{proof}
Proceeding as in the proof of Theorem~3.7 in~\citep{Cesa-BianchiL06} we only need to show that
\[
\frac{1}{2} E\left[ \max_{1 \le i \le d} Z_i\right] \ge f(n,d),
\]
where $Z_i= 2 X_i - n$ and $X_i \sim B(n, \frac{1}{2})$. We simply do it through Theorem~\ref{theorem:max_bin}.
For the second statement, we use the fact that $\lim_{n \to \infty} \psi\left(\frac{1.6 \sqrt{\log d}}{2 \sqrt{n}}\right) = 1$.
\end{proof}

The theorem proves a non-asymptotic lower bounds, while at the same time
recovering the optimal constant of the asymptotic one in
\citet{Cesa-BianchiL06}. Also, differently from the asymptotic lower bound in
\citet{Cesa-BianchiL06}, we do not need to take the limit for $n$ that goes to
infinity.

\bibliographystyle{plainnat}
\bibliography{learning}

\appendix

\section{Proof of Theorem~\ref{theorem:max_bin}}
\begin{proof}
We use Theorem~2 in \cite{McKay1989}, that specialized to our case says that
\begin{equation}
\label{equation:bin-1}
\Pr \left( Z \ge  \frac{1}{2} n + k  \right)
\ge \sqrt{n} \binom{n-1}{ \frac{1}{2} n + k -1} 2^{-n} \frac{Q(y)}{\phi(y)},
\end{equation}
where $\phi(x)$ is the unit variance, zero mean Gaussian, $\frac{1}{\sqrt{2 \pi}} \exp(-\frac{x^2}{2})$ and $Q(x)$ is its CDF, $\int_{x}^{+\infty} \phi(u) du$.

We lower bounding the ratio $\frac{Q(y)}{\phi(y)}$ using Lemma~\ref{lemma:boyd}:
\[
\frac{Q(y)}{\phi(y)}
= \exp\left(\frac{x^2}{2}\right) \int_{x}^{+\infty} \exp\left(-\frac{t^2}{2}\right) dt
\ge \frac{\pi}{(\pi-1)x+\sqrt{x^2+2 \pi}}.
\]

To bound the binomial coefficient we make use of the following Stirling approximation, for any $n\ge 1$,
\[
\sqrt{2 \pi n} n^n \exp(-n) < n! < \exp\left(\frac{1}{12}\right)\sqrt{2 \pi n} n^n \exp(-n)~.
\]
Hence, for any $n \ge 2$ and $1\le q \le n-1$, after some algebra we obtain
\begin{align*}
{n \choose q}
& \ge \frac{1}{\exp\left(\frac{1}{6}\right) \sqrt{2 \pi}} \left(\frac{n}{n-q}\right)^{n-q} \left(\frac{n}{q}\right)^{q} \sqrt{\frac{n}{q(n-q)}} \\
& \ge \frac{1}{\exp\left(\frac{1}{6}\right) \sqrt{2 \pi}} 2^n \exp\left(-n D\left(\frac{q}{n} \| \frac{1}{2}\right)\right) \sqrt{\frac{n}{q(n-q)}}.
\end{align*}
where in the equality we used the definition of $D$.
Also, we have
\begin{equation}
\label{equation:bin-3}
{n-1 \choose \frac{1}{2} n + k - 1} = {n \choose \frac{1}{2} n + k} \left(\frac{1}{2} + \frac{k}{n}\right) .
\end{equation}
Putting together \eqref{equation:bin-1}-\eqref{equation:bin-3}, and using the definition of $y$ we have
\begin{align*}
&\Pr\left[ Z \ge \frac{1}{2} n + k \right]
\ge \frac{1}{\exp\left(\frac{1}{6}\right) \sqrt{2 \pi}} \exp\left(-n D\left(\frac{1}{2}+\frac{k}{n} \| \frac{1}{2}\right)\right) \sqrt{\frac{\frac{1}{2} + \frac{k}{n}}{\frac{1}{2}-\frac{k}{n}}}  \frac{Q(y)}{\phi(y)} \\
&\ge \frac{1}{\exp\left(\frac{1}{6}\right) \sqrt{2 \pi}} \exp\left(-n D\left(\frac{1}{2}+\frac{k}{n} \| \frac{1}{2}\right)\right) \frac{\pi}{(\pi-1)y+\sqrt{y^2+2 \pi}}. \qedhere
\end{align*}
\end{proof}

\end{document}
