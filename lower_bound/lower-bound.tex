\documentclass{article}

\usepackage{algorithm, algorithmic}
\usepackage{amsmath, amsthm, amssymb}
\usepackage{fullpage}
\usepackage{natbib}
\usepackage{times}

\DeclareMathOperator*{\Exp}{\mathbf{E}}
\DeclareMathOperator*{\argmin}{arg\,min}
\DeclareMathOperator*{\argmax}{arg\,max}
\DeclareMathOperator{\Regret}{Regret}

\newcommand{\field}[1]{\mathbb{#1}}
\newcommand{\R}{\field{R}}
\newcommand{\Nat}{\field{N}}
\newcommand{\Var}{\mathrm{Var}}

\newtheorem{theorem}{Theorem}
\newtheorem{lemma}[theorem]{Lemma}
\newtheorem{corrollary}[theorem]{Corollary}
\newtheorem{remark}[theorem]{Remark}
\newtheorem{proposition}[theorem]{Proposition}

\newcommand{\reals}{\mathbb{R}}
\newcommand{\sign}{{\rm sign}}

\begin{document}

\title{Optimal Non-Asymptotic Lower Bound \\ on the \\ Minimax Regret for Learning with Expert Advice}
\author{
\begin{tabular}{c@{\hskip 1in}c}
  Francesco Orabona & David Pal \\
  francesco@orabona.com & dpal@yahoo-inc.com \\
\end{tabular}
\\\\
Yahoo Labs \\
New York, NY, USA
}


\maketitle

\begin{abstract}
We prove non-asymptotic lower bounds on the expectation of the maximum of $d$
independent Gaussian variables and expectation of the maximum of $d$
independent symmetric random walks. Both lower bound recovers
the optimal leading constant in the limit.

A simple application of the lower bound for random walks is a non-asymptotic lower
bound on the minimax regret of online learning with expert advice.
\end{abstract}

\section{Introduction}

Let $X_1, X_2, \dots, X_d$ be i.i.d. Gaussian random variables $N(0,\sigma^2)$.
It easy to prove that (see Appendix~\ref{section:upper-bounds})
\begin{equation}
\label{equation:upper-bound-on-maximum-of-gaussians}
\Exp \left[ \max_{1 \le i \le d} X_i \right] \le \sigma \sqrt{2 \ln d} \qquad \text{for any $d \ge 1$} \; .
\end{equation}
It is also well known that
\begin{equation}
\label{equation:limit-maximum-of-gaussians}
\lim_{d \to \infty} \frac{\Exp \left[ \max_{1 \le i \le d} X_i \right]}{\sigma \sqrt{2 \ln d}} = 1 \; .
\end{equation}
In section~\ref{section:maximum-of-gaussians}, we prove a non-asymptotic
$\Omega(\sigma \sqrt{\log d})$ lower bound on $\Exp[\max_{1 \le i \le d} X_i]$. The leading term
of the lower bound is asymptotically $\sqrt{2 \ln d}$. In other words, the lower bound implies~\eqref{equation:limit-maximum-of-gaussians}.

Discrete analog of a Gaussian random variable is the symmetric random walk. Recall that a random walk $Z^{(n)}$
of length $n$ is a sum $Z^{(n)} = Y_1 + Y_2 + \dots + Y_n$ of $n$ i.i.d. Rademacher variables, which have probability distribution
$\Pr[Y_i = +1] = \Pr[Y_i = -1] = 1/2$. We consider $d$ independent symmetric random walks $Z^{(n)}_1, Z^{(n)}_2, \dots, Z^{(n)}_d$ of length $n$.
Analogously to \eqref{equation:upper-bound-on-maximum-of-gaussians}, it is easy to prove that (see Appendix~\ref{section:upper-bounds})
\begin{equation}
\label{equation:upper-bound-on-maximum-of-random-walks}
\Exp \left[ \max_{1 \le i \le d} Z^{(n)}_i \right] \le \sqrt{2 n \ln d} \qquad \text{for any $n \ge 0$ and any $d \ge 1$}\; .
\end{equation}
Note that $\sigma^2$ in \eqref{equation:upper-bound-on-maximum-of-gaussians} is replaced by $\Var(Z^{(n)}_i) = n$. By central limit theorem $\frac{Z^{(n)}_i}{\sqrt{n}}$
as $n \to \infty$ converges in distribution to $N(0,1)$. From this fact, it possible to prove
the analog of \eqref{equation:limit-maximum-of-gaussians},
\begin{equation}
\label{equation:limit-maximum-of-random-walks}
\lim_{d \to \infty} \lim_{n \to \infty} \frac{\Exp\left[ \max_{1 \le i \le d} Z^{(n)}_i \right]}{\sqrt{2 n \ln d}} = 1 \; .
\end{equation}
We prove a non-asymptotic $\Omega(\sqrt{n \log d})$ lower bound on $\Exp\left[ \max_{1 \le i \le d} Z^{(n)}_i \right]$.
Same as for the Gaussian case, the leading term of the lower bound is asymptotically $\sqrt{2 n \ln d}$
matching~\eqref{equation:limit-maximum-of-random-walks}.

In section~\ref{section:experts}, we show a simple application of the lower
bound on $\Exp\left[\max_{1 \le i \le d} Z^{(n)}_i \right]$ to the problem of learning with
expert advice.  This problem was extensively studied in the online learning
literature; see~\citep{Cesa-BianchiL06}.  Our bound is optimal in the sense
that for large $d$ and large $n$ it recovers the right leading constant.

\section{Maximum of Gaussians}
\label{section:maximum-of-gaussians}

It is well known that the maximum of Gaussian variables converges to a Gumbel
distribution. Here we quantify the rate of convergence for the expectation of
maximum of independent Gaussian variables.

We start with a lower bound on the Mill's ratio of Gaussian variables. Mill's ratio of a random variable $X$
with density function $f(x)$ is the ratio $\frac{\Pr[X > x]}{f(x)}$. For standard Gaussian
$N(0,1)$ with density $\phi(x) = \frac{1}{\sqrt{2 \pi}} \exp\left( - \frac{x^2}{2} \right)$
and $\Phi(x) = \int_{-\infty}^x \frac{1}{\sqrt{2 \pi}} \exp\left( - \frac{t^2}{2} \right) dt$,
the Mill's ratio is $\frac{1 - \Phi(x)}{\phi(x)}$. Bounds on Mill's ratio provide bounds on $\Pr[X > x]$.
In this paper we rely on a lower bound on Mill's ratio by~\cite{Boyd-1959}.

\begin{lemma}[Mill's ratio for standard Gaussian~\citep{Boyd-1959}]
\label{lemma:boyd}
For any $x \ge 0$,
$$
\frac{1 - \Phi(x)}{\phi(x)}
= \exp\left(\frac{x^2}{2}\right) \int_x^{\infty} \exp\left(-\frac{t^2}{2}\right) dt
\ge \frac{\pi}{(\pi-1)x+\sqrt{x^2+2 \pi}}.
$$
\end{lemma}

The bound on Mill's ratio can be turned into a lower bound on the tail of the distribution.

\begin{corrollary}[Lower Bound on Gaussian Tail]
Let $X \sim N(0, \sigma^2)$ and $x \ge 0$. Then,
$$
\Pr[X \ge x] \ge \exp\left(-\frac{x^2}{2 \sigma^2}\right) \frac{1}{\sqrt{2\pi}\frac{x}{\sigma}+2}.
$$
\end{corrollary}

\begin{proof}
We have
\begin{align*}
\Pr[X \ge x]
& = \frac{1}{\sigma \sqrt{2 \pi}} \int_x^{\infty} \exp \left( -\frac{t^2}{2 \sigma^2} \right) d t \\
& = \frac{1}{\sqrt{2 \pi}} \int_\frac{x}{\sigma}^{\infty} \exp \left( -\frac{t^2}{2} \right) d t \\
& \ge \frac{1}{\sqrt{2 \pi}} \exp\left(-\frac{x^2}{2 \sigma^2}\right) \frac{\pi}{(\pi-1)\frac{x}{\sigma}+\sqrt{\frac{x^2}{\sigma^2}+2 \pi}} & \text{(by Lemma~\ref{lemma:boyd})} \\
& \ge \frac{1}{\sqrt{2}} \exp\left(-\frac{x^2}{2 \sigma^2}\right) \frac{\sqrt{\pi}}{\pi\frac{x}{\sigma}+\sqrt{2 \pi}} \\
& = \exp\left(-\frac{x^2}{2 \sigma^2}\right) \frac{1}{\sqrt{2\pi}\frac{x}{\sigma}+2} \; .
\end{align*}
\end{proof}

\begin{theorem}[Lower Bound on Maximum of Independent Gaussians]
Let $X_1, X_2, \dots, X_d$ be independent Gaussian random variables $N(0,\sigma^2)$. For any $d \ge 2$,
\begin{align}
\Exp \left[\max_{1 \le i \le d} X_i\right]
& \ge \sigma \left(1 - \exp\left(-\frac{\sqrt{\ln d}}{6.35}\right)\right) \left(\sqrt{2 \ln d - 2 \ln \ln d} +\sqrt\frac{2}{\pi}\right) -\sqrt{\frac{2}{\pi}} \sigma \label{equation:maximum-of-gaussians-lower-bound-1} \\
& \ge 0.13 \sigma \sqrt{\ln d} - 0.7 \sigma \label{equation:maximum-of-gaussians-lower-bound-2} \; .
\end{align}
\end{theorem}

\begin{proof}
Let $A$ be the event that at least one of the $X_i$ is greater than $C \sigma
\sqrt{\ln d}$ where $C = C(d) = \sqrt{2 - \frac{2 \ln \ln d}{\ln d}}$. We
denote by $\overline{A}$ the complement of this event. We have
\begin{align}
\Exp \left[ \max_{1 \le i \le d} X_i \right]
& = \Exp \left[ \max_{1 \le i \le d} X_i ~ \middle|~ A \right] \cdot \Pr[A] + \Exp \left[ \max_{1 \le i \le d} X_i ~\middle|~ \overline{A} \right] \cdot \Pr \left[ \overline{A} \right] \notag \\
& \ge \Exp \left[ \max_{1 \le i \le d} X_i ~\middle|~ A \right] \cdot \Pr[A] + \Exp \left[ X_1 ~\middle|~ \overline{A} \right] \cdot \Pr[\overline{A}] \notag \\
& = \Exp \left[ \max_{1 \le i \le d} X_i~\middle|~ A \right] \cdot \Pr[A] + \Exp \left[ X_1~|~ X_1 \le C \sigma \sqrt{\ln d} \right] \cdot \Pr[\overline{A}] \notag \\
& \ge \Exp \left[ \max_{1 \le i \le d} X_i~\middle|~ A \right] \cdot \Pr[A] + \Exp[X_1~|~ X_1 \le 0] \cdot \Pr[\overline{A}] \notag \\
& \ge \Exp \left[ \max_{1 \le i \le d} X_i~\middle|~ A \right] \cdot \Pr[A] - \sigma \sqrt{\frac{2}{\pi}} \cdot \Pr[\overline{A}] \notag \\
& \ge C \sigma \sqrt{\ln d} \cdot \Pr[A] - \sigma \sqrt{\frac{2}{\pi}} (1 - \Pr[A]) \notag \\
& = \sigma \left(C\sqrt{\ln d} + \sqrt{\frac{2}{\pi}}\right) \Pr[A] -  \sigma \sqrt{\frac{2}{\pi}} \label{equation:maximum-of-gaussians-lower-bound-3}
\end{align}
where we used that $\Exp[X_1 ~|~ X_1 \le 0] = \frac{1}{\Pr[X_1 \le 0]} \int_{-\infty}^0 \frac{x}{\sigma \sqrt{2\pi}} \exp \left(- \frac{x^2}{2\sigma^2} \right) = - \sigma \sqrt{\frac{2}{\pi}}$.

It remains to lower bound $\Pr[A]$, which we do as follows
\begin{align}
\Pr[A]
& = 1 - \Pr[\overline{A}] \notag \\
& = 1 - \prod_{i=1}^d \Pr \left[ X_i \le C \sigma \sqrt{\ln d} \right]  \notag \\
& = 1 - \left(1 - \Pr\left[ X_1 > C \sigma \sqrt{\ln d} \right] \right)^d \notag \\
& \ge 1 - \exp\left(-d \cdot \Pr\left[X_1 \ge C \sigma \sqrt{\ln d} \right]\right) \notag \\
& \ge 1 - \exp\left(-d \exp\left(-\frac{C^2 \ln d}{2}\right) \frac{1}{\sqrt{2\pi}C \sqrt{\ln d}+2} \right) \notag \\
& = 1 - \exp\left(-\frac{d^{1-\frac{C^2}{2}}}{C \sqrt{2\pi \ln d}+2}\right) \label{equation:maximum-of-gaussians-lower-bound-4} \; .
\end{align}
where in the first inequality we used the elementary inequality $1 - x \le \exp(-x)$ valid for all $x \in \R$.

Since $C = \sqrt{2 - \frac{2 \ln \ln d}{\ln d}}$ we have $d^{1-\frac{C^2}{2}} = \ln d$. Subtistuting this into \eqref{equation:maximum-of-gaussians-lower-bound-4}, we get
\begin{equation}
\label{equation-maximum-of-gaussians-lower-bound-5}
\Pr[A] \ge 1 - \exp\left(-\frac{\ln d}{C \sqrt{2\pi \ln d}+2}\right) = 1 - \exp\left(-\frac{\sqrt{\ln d}}{C \sqrt{2\pi}+2}\right) \; .
\end{equation}
The function $C(d)$ is decreasing on the interval $[1,e^e]$, increasing on $[e^e, \infty)$, and $\lim_{d \to \infty} C(d) = \sqrt{2}$. From these properties
we can deduce that $C(d) \le \max\{C(2), \sqrt{2}\} \le 1.75$ for any $d \in [2,\infty)$. Therefore, $C\sqrt{2 \pi} + 2 \le 6.35$ and hence
\begin{equation}
\label{equation:maximum-of-gaussians-lower-bound-6}
\Pr[A] \ge 1 - \exp\left(-\frac{\sqrt{\ln d}}{6.35}\right) \; .
\end{equation}
Inequalities \eqref{equation:maximum-of-gaussians-lower-bound-3} and \eqref{equation:maximum-of-gaussians-lower-bound-6} together imply bound \eqref{equation:maximum-of-gaussians-lower-bound-1}.

Bound \eqref{equation:maximum-of-gaussians-lower-bound-2} is obtained from \eqref{equation:maximum-of-gaussians-lower-bound-1} by noticing that
\begin{align*}
& \sigma \left(1 - \exp\left(-\frac{\sqrt{\ln d}}{6.35}\right)\right) \left(\sqrt{2 \ln d - 2 \ln \ln d} +\sqrt\frac{2}{\pi}\right) -\sqrt{\frac{2}{\pi}} \sigma \\
& = \sigma \left(1 - \exp\left(-\frac{\sqrt{\ln d}}{6.35}\right)\right) \sqrt{2 \ln d - 2 \ln \ln d} - \exp\left(-\frac{\sqrt{\ln d}}{6.35}\right) \sqrt{\frac{2}{\pi}} \sigma \\
& \ge 0.1227 \cdot \sigma \sqrt{2 \ln d - 2 \ln \ln d} - 0.7 \sigma \\
& = 0.1227 \cdot \sigma \sqrt{\ln d} \cdot C(d) - 0.7 \sigma
\end{align*}
where we used that $\exp\left(-\frac{\sqrt{\ln d}}{6.35}\right) \le 0.8773$ for any $d \ge 2$.
Since $C(d)$ has minimum at $d = e^e$, it follows that $C(d) \ge C(e^e) = \sqrt{2 - \frac{2}{e}} \ge 1.1243$ for any $d \ge 2$.
\end{proof}

\section{Binomial Case}
\label{section:maximum-of-random-walks}

In the Binomial case, we expect the discrete nature of the variable to play a
role. In fact, we expect it to behave like a Gaussian only when the number of
draws goes to infinity.  We quantify the intuition with the function $\psi:[-\frac{1}{2},\frac{1}{2}] \to \R$
defined as
$$
\psi(x) = \frac{D \left(\frac{1}{2}+x \middle\| \frac{1}{2} \right)}{2 x^2} \; .
$$
where $D(p\|q)$ is the Kullback-Leibler divergence between $\text{Bernoulli}(p)$ and $\text{Bernoulli}(q)$
defined as
$$
D(p\|q) = p \ln \frac{p}{q}+(1-p) \ln\frac{1-p}{1-q}
$$
The function $\psi(x)$ satisfies the following properties
\begin{itemize}
\item $\psi(x) = \psi(-x)$
\item $\psi(x)$ is increasing for $0\le x \le \frac{1}{2}$.
\item $\psi(0) = 1$
\item $\psi(0.5) = 2 \ln(2) \approx 1.3863$
\end{itemize}

As in the Gaussian case, for the Binomial case we need lower bound on the
probability tail.  We will use the next Theorem from \cite{nOrabona13}, whose
proof is the Appendix for completeness.

\begin{theorem}
\label{theorem:binomial}
Let $n \ge 2$ an even number and $Z$ a Binomial random variable
$B(n,\frac{1}{2})$. Then for any $k \in \Nat_0$ such that $k\le
\frac{1}{2}n-1$, we have
$$
\Pr \left[ Z \ge \frac{1}{2} n + k\right]
\ge \frac{\exp\left(-n D(\frac{1}{2}+\frac{k}{n}\|\frac{1}{2})\right)}{2 \exp\left(\frac{1}{6}\right)} \frac{\sqrt{2 \pi}}{(\pi-1)y+\sqrt{y^2+2 \pi}} \; ,
$$
where $y=\frac{2 k}{\sqrt{n}}$.
\end{theorem}

\begin{corrollary}
Let $Z \sim B(n, 1/2)$ and $k \ge 1$ and $n \ge 2$ even. Then
$$
\Pr \left[ Z \ge \frac{1}{2} n + k-1 \right] \ge \exp\left(-\frac{1}{6}\right) \exp\left(- 2 \psi\left(\frac{k}{n}\right) \frac{k^2}{n} \right) \frac{1}{\sqrt{2\pi} \frac{2 k}{\sqrt{n}} + 2 }~.
$$
\end{corrollary}

\begin{proof}
\begin{align*}
\Pr \left[ Z \ge  \frac{1}{2} n + k-1 \right]
& = \Pr \left[ Z \ge \frac{1}{2} n + \lceil k - 1\rceil \right] \\
& \ge \frac{\exp\left(-n D(\frac{1}{2}+\frac{\lceil k -1\rceil}{n} \| \frac{1}{2})\right)}{2 \exp\left(\frac{1}{6}\right)} \frac{\sqrt{2 \pi}}{(\pi-1)\frac{2\lceil k -1\rceil}{\sqrt{n}}+\sqrt{\left(\frac{2\lceil k -1\rceil}{\sqrt{n}}\right)^2+2 \pi}} \\
& \ge \frac{\exp\left(-n D(\frac{1}{2}+\frac{k}{n} \| \frac{1}{2})\right)}{2 \exp\left(\frac{1}{6}\right)} \frac{\sqrt{2 \pi}}{(\pi-1)\frac{2k}{\sqrt{n}}+\sqrt{\left(\frac{2k}{\sqrt{n}}\right)^2+2 \pi}} \\
& = \frac{\exp\left(- 2 \psi(\frac{k}{n}) \frac{k^2}{n} \right)}{2 \exp\left(\frac{1}{6}\right)} \frac{\sqrt{2 \pi}}{(\pi-1)\frac{2k}{\sqrt{n}}+\sqrt{\left(\frac{2k}{\sqrt{n}}\right)^2+2 \pi}} \\
& \ge \frac{\exp\left(- 2 \psi(\frac{k}{n}) \frac{k^2}{n} \right)}{2 \exp\left(\frac{1}{6}\right)} \frac{\sqrt{2 \pi}}{(\pi-1)\frac{2k}{\sqrt{n}}+\sqrt{\left(\frac{2k}{\sqrt{n}}\right)^2+2 \pi}} \\
& \ge \exp\left(-\frac{1}{6}\right) \exp\left(- 2 \psi\left(\frac{k}{n}\right) \frac{k^2}{n} \right) \frac{1}{\sqrt{2\pi} \frac{2 k}{\sqrt{n}} + 2} \; ,
\end{align*}
where in the second equality we used Theorem~\ref{theorem:binomial}.
\end{proof}

\begin{theorem}
\label{theorem:maximum-of-random-walks}
Let $Z^{(n)}_1, Z^{(n)}_2, \dots, Z^{(n)}_d$ be $d$ independent symmetric random walks of length $n$. If $2 \le d \le \exp(\frac{n}{4})$ and $n\ge 10$ is even,
\begin{align*}
\Exp \left[ \max_{1 \le i \le d} Z^{(n)}_i \right]
& \ge \frac{1}{\sqrt{\psi\left(\frac{1.6\sqrt{\ln d}}{2 \sqrt{n}}\right)}}\sqrt{n}\left(1 - \exp\left(-\frac{\sqrt{\ln d}}{3.1 \sqrt{2\pi}}\right)\right) \left(\sqrt{2 \ln d -\ln \ln d}-1\right) -\sqrt{n} \\
& \ge 0.13 \sqrt{n \ln d} - 2 \sqrt{n}.
\end{align*}
\end{theorem}

\begin{proof}
Define $Z^{(n)}_i = 2 B_i-n$.  Define the event $A$ equal to the case that at least
one of the $X_i$ is greater or equal than $C \sqrt{n \ln d}-2$, where
$C=\frac{1}{\sqrt{\psi\left(\frac{\sqrt{\ln d}}{2
\sqrt{n}}\right)}}\sqrt{2-\frac{\ln \ln d}{\ln d}}$. Also, the function
$f(d)=\sqrt{2-\frac{\ln \ln d}{\ln d}}$ has a minimum in $d=e^e$, hence
$1.08 \le \frac{f(15)}{\sqrt{2 \ln 2}} \le C\le f(2)\le 1.6$.

Notice that the condition on $n$ and $d$ assures that $\frac{C \sqrt{n \ln d}}{2}>1$ and $\frac{C \sqrt{n \ln d}}{2}\le \frac{1}{2} n - 1$.
\begin{align*}
\Exp \left[ \max_{1 \le i \le d} Z^{(n)}_i \right]
& = \Exp \left[ \max_{1 \le i \le d} Z^{(n)}_i ~ \middle|~ A \right] \cdot \Pr[A] + \Exp \left[ \max_{1 \le i \le d} Z^{(n)}_i ~\middle|~ \overline{A} \right] \cdot \Pr \left[ \overline{A} \right] \\
& \ge \Exp \left[ \max_{1 \le i \le d} Z^{(n)}_i ~\middle|~ A \right] \cdot \Pr[A] + \Exp \left[ \max_{1 \le i \le d} Z^{(n)}_i ~\middle|~ \overline{A} \right] \cdot \Pr \left[ \overline{A} \right]\\
& \ge \Exp \left[ \max_{1 \le i \le d} Z^{(n)}_i ~\middle|~ A \right] \cdot \Pr[A] + \Exp\left[ Z^{(n)}_1 ~\middle|~ \overline{A} \right] \cdot \Pr \left[ \overline{A} \right] \\
& = \Exp \left[ \max_{1 \le i \le d} Z^{(n)}_i ~\middle|~ A \right] \cdot \Pr[A] + \Exp \left[ Z^{(n)}_1 ~\middle|~ Z^{(n)}_1 \le C \sigma \sqrt{\ln d} \right] \cdot \Pr \left[ \overline{A} \right]\\
& \ge \Exp \left[ \max_{1 \le i \le d} Z^{(n)}_i ~\middle|~ A \right] \cdot \Pr[A] + \Exp \left[ Z^{(n)}_1 ~\middle|~ Z^{(n)}_1 \le 0 \right] \cdot \Pr \left[ \overline{A} \right] \\
& \ge (C \sigma \sqrt{n \ln d} - 2) \Pr[A] + \Exp \left[ Z^{(n)}_1 ~\middle|~ Z^{(n)}_1 < 0 \right] (1 - \Pr[A]) \; .
\end{align*}

First, we lower bound $\Exp \left[ Z^{(n)}_1 ~\middle|~ Z^{(n)}_1 \le 0 \right]$. Using the fact that $Z^{(n)}_1$ is symmetric and has zero mean, we have
\begin{align*}
\Exp \left[ Z^{(n)}_1 ~\middle|~ Z^{(n)}_1 \le 0 \right]
& = \sum_{k=-n}^0 k \cdot \Pr[Z^{(n)}_1 = k ~|~ Z^{(n)}_1 \le 0] \\
& = \frac{1}{\Pr[Z^{(n)}_1 \le 0]} \sum_{k=-n}^0 k \cdot \Pr[Z^{(n)}_1 = k] \\
& \ge 2 \sum_{k=-n}^0 k \cdot \Pr[Z^{(n)}_1 = k] \\
& = - \sum_{k=-n}^n |k| \cdot \Pr[Z^((n))_1 = k] \\
& = - \Exp[|Z^{(n)}_1|] \\
& = - \Exp \left[ \sqrt{ \left( Z^{(n)}_1 \right)^2} \right] \\
& \ge - \sqrt{\Exp \left[ \left( Z^{(n)}_1 \right)^2 \right]} \\
& = -\sqrt{n}.
\end{align*}

Now let's focus on $\Pr[A]$. As in the Gaussian case, we can lower bound it as
\begin{align*}
\Pr[A]
& = 1 - \Pr \left[ Z^{(n)}_1 < C \sqrt{n \ln d} - 2 \right]^d \\
& = 1 - \left(1 - \Pr\left[Z^{(n)}_1\ge C \sqrt{n \ln d}-2 \right] \right)^d \\
& = 1 - \left(1 - \Pr\left[B_1 \ge \frac{C \sqrt{n \ln d}}{2} +\frac{n}{2} - 1 \right] \right)^d \\
& \ge 1-\exp\left(-d \cdot \Pr \left[ B_1\ge \frac{C \sqrt{n \ln d}}{2} +\frac{n}{2}-1 \right] \right) \\
& \ge 1 - \exp\left(-\frac{\exp\left(-\frac{1}{6}\right) d^{1-\frac{C^2}{2} \psi\left(\frac{C \sqrt{\ln d}}{2 \sqrt{n}}\right)}}{C \sqrt{2\pi} \sqrt{\ln d}+2}\right) \\
& \ge 1 - \exp\left(-\frac{\exp\left(-\frac{1}{6}\right) d^{1-\frac{C^2}{2} \psi\left(\frac{1.6 \sqrt{\ln d}}{2 \sqrt{n}}\right)}}{1.6 \sqrt{2\pi} \sqrt{\ln d}+2}\right).
%
% &\ge 1-\left(1-\exp\left(- \psi\left(\frac{C \sqrt{\ln d}}{2 \sqrt{n}}\right) \frac{C^2 \ln d}{2}\right) \frac{\exp\left(-\frac{1}{6}\right)}{\sqrt{2\pi}C \sqrt{\ln d}+2}\right)^d \\
% &\ge 1-\left(1-\exp\left(- \psi\left(\frac{1.6 \sqrt{\ln d}}{2 \sqrt{n}}\right) \frac{C^2 \ln d}{2}\right) \frac{\exp\left(-\frac{1}{6}\right)}{\sqrt{2\pi}C \sqrt{\ln d}+2}\right)^d \\
% &= 1-\left(1-\frac{\exp\left(-\frac{1}{6}\right) d^{-\frac{C^2}{2} \psi\left(\frac{1.6 \sqrt{\ln d}}{2 \sqrt{n}}\right)}}{\sqrt{2\pi}C \sqrt{\ln d}+2}\right)^d \\
% &= 1- \exp\left(d \ln\left(1-\frac{\exp\left(-\frac{1}{6}\right) d^{-\frac{C^2}{2} \psi\left(\frac{1.6 \sqrt{\ln d}}{2 \sqrt{n}}\right)}}{\sqrt{2\pi}C \sqrt{\ln d}+2}\right)\right) \\
% &\ge 1- \exp\left(d \ln \left(1-\frac{\exp\left(-\frac{1}{6}\right) d^{-\frac{C^2}{2} \psi\left(\frac{1.6 \sqrt{\ln d}}{2 \sqrt{n}}\right)}}{1.6 \sqrt{2\pi} \sqrt{\ln d}+2}\right)\right) \\
% &\ge 1 - \exp\left(-\frac{\exp\left(-\frac{1}{6}\right) d^{1-\frac{C^2}{2} \psi\left(\frac{1.6 \sqrt{\ln d}}{2 \sqrt{n}}\right)}}{1.6 \sqrt{2\pi} \sqrt{\ln d}+2}\right).
\end{align*}
%where in the last inequality we used the elementary inequality $\ln(1-\frac{1}{x}) \le -\frac{1}{x}, \forall  0\le x>1$.
We now use the fact that $C=\frac{1}{\sqrt{\psi\left(\frac{1.6 \sqrt{\ln d}}{2 \sqrt{n}}\right)}}\sqrt{2- \frac{\ln \ln d}{\ln d}}$ that implies $d^{1-\frac{C^2}{2} \psi\left(\frac{1.6 \sqrt{\ln d}}{2 \sqrt{n}}\right)}=\ln d$. Hence, we obtain
\begin{align*}
\Pr[A]
& \ge 1 - \exp\left(-\frac{\exp\left(-\frac{1}{6}\right) d^{1-\frac{C^2}{2} \psi\left(\frac{1.6 \sqrt{\ln d}}{2 \sqrt{n}}\right)}}{1.6 \sqrt{2\pi} \sqrt{\ln d}+2}\right) \\
& = 1 - \exp\left(-\frac{\exp\left(-\frac{1}{6}\right) \ln d}{1.6 \sqrt{2\pi} \sqrt{\ln d}+2}\right) \\
& \ge 1 - \exp\left(-\frac{\exp\left(-\frac{1}{6}\right) \sqrt{\ln d}}{2.6 \sqrt{2\pi}}\right)\\
& \ge 1 - \exp\left(-\frac{\sqrt{\ln d}}{3.1 \sqrt{2\pi}}\right),
\end{align*}
where in the last equality we used the fact that $\sqrt{2\pi} \sqrt{\ln d} > 2$ for $d\ge 2$.

Putting all together, we have the stated bound.
\end{proof}

\section{Learning with Expert Advice}
\label{section:experts}

We present here our main result.

\begin{theorem}
Let the outcome space $\mathcal{Y}=\{0,1\}$, and the decision space
$\mathcal{D}=[0,1]$, and $\ell$ the absolute loss, $\ell(p,q)=|p-q|$. Let $n
\ge 4$ and even, and $2\le d \le \exp(\frac{n}{4})$.  Define
\[
f(n,d)=\frac{1}{2}\frac{1}{\sqrt{\psi\left(\frac{1.6 \sqrt{\ln d}}{2 \sqrt{n}}\right)}}\sqrt{n}\left(1 - \exp\left(-\frac{\sqrt{\ln d}}{3.1 \sqrt{2\pi}}\right)\right) \left(\sqrt{2 \ln d - \ln \ln d}-1\right) -\frac{1}{2}\sqrt{n}~.
\]
Then
\[
\Regret^{(d)}(n)\ge f(n,d)
\]
and
\[
\sup_{n,d} \frac{f(n,d)}{\sqrt{\frac{n}{2} \ln d}} \ge 1~.
\]
\end{theorem}
%
\begin{proof}
Proceeding as in the proof of Theorem~3.7 in~\citep{Cesa-BianchiL06} we only need to show that
\[
\frac{1}{2} \Exp \left[ \max_{1 \le i \le d} Z_i\right] \ge f(n,d),
\]
where $Z_i= 2 X_i - n$ and $X_i \sim B(n, \frac{1}{2})$. We simply do it through Theorem~\ref{theorem:maximum-of-random-walks}.
For the second statement, we use the fact that $\lim_{n \to \infty} \psi\left(\frac{1.6 \sqrt{\ln d}}{2 \sqrt{n}}\right) = 1$.
\end{proof}

The theorem proves a non-asymptotic lower bounds, while at the same time
recovering the optimal constant of the asymptotic one in
\citet{Cesa-BianchiL06}. Also, differently from the asymptotic lower bound in
\citet{Cesa-BianchiL06}, we do not need to take the limit for $n$ that goes to
infinity.

\bibliographystyle{plainnat}
\bibliography{biblio}

\appendix

\section{Proof of Theorem~\ref{theorem:binomial}}

\begin{proof}
We use Theorem~2 in \cite{McKay1989}, that specialized to our case says that
\begin{equation}
\label{equation:bin-1}
\Pr \left[ Z \ge  \frac{1}{2} n + k  \right] \ge \sqrt{n} \binom{n-1}{ \frac{1}{2} n + k -1} 2^{-n} \frac{1 - \Phi(y)}{\phi(y)},
\end{equation}
where $\phi(x) = \frac{1}{\sqrt{2 \pi}} \exp(-\frac{x^2}{2})$ is the density of $N(0,1)$ and
$\Phi(x) = \int_{-\infty}^x \phi(t) dt$ is its cumulative density function.

We use Lemma~\ref{lemma:boyd} to lower bound the Mill's ratio $\frac{1 - \Phi(y)}{\phi(y)}$,
$$
\frac{1 - \Phi(y)}{\phi(y)}
= \exp\left(\frac{x^2}{2}\right) \int_{x}^{+\infty} \exp\left(-\frac{t^2}{2}\right) dt
\ge \frac{\pi}{(\pi-1)x+\sqrt{x^2+2 \pi}}.
$$

We bound the binomial coefficient in \eqref{equation:bin-1} by using Stirling's formula for the factorial.
We use explicit upper and lower bounds due to~\cite{Robbins-1955} valid for any $n\ge 1$,
$$
\sqrt{2 \pi n} \left( \frac{n}{e} \right)^n < n! < \exp\left(\frac{1}{12}\right) \sqrt{2 \pi n} \left( \frac{n}{e} \right)^n \; .
$$
Hence, for any $n \ge 2$ and $1\le q \le n-1$, after some algebra we obtain
\begin{align*}
\binom{n}{q}
& = \frac{n!}{q! (n-q)!} \\
& \ge \frac{1}{\exp\left(\frac{1}{6}\right) \sqrt{2 \pi}} \left(\frac{n}{n-q}\right)^{n-q} \left(\frac{n}{q}\right)^{q} \sqrt{\frac{n}{q(n-q)}} \\
& = \frac{1}{\exp\left(\frac{1}{6}\right) \sqrt{2 \pi}} 2^n \exp\left(-n D\left(\frac{q}{n} \middle\| \frac{1}{2}\right)\right) \sqrt{\frac{n}{q(n-q)}}.
\end{align*}
where in the equality we used the definition of $D$.
Also, we have
\begin{equation}
\label{equation:bin-3}
{n-1 \choose \frac{1}{2} n + k - 1} = {n \choose \frac{1}{2} n + k} \left(\frac{1}{2} + \frac{k}{n}\right) .
\end{equation}
Putting together \eqref{equation:bin-1}-\eqref{equation:bin-3}, and using the definition of $y$ we have
\begin{align*}
\Pr\left[ Z \ge \frac{1}{2} n + k \right]
& \ge \frac{1}{\exp\left(\frac{1}{6}\right) \sqrt{2 \pi}} \exp\left(-n D\left(\frac{1}{2}+\frac{k}{n} \middle\| \frac{1}{2}\right)\right) \sqrt{\frac{\frac{1}{2} + \frac{k}{n}}{\frac{1}{2}-\frac{k}{n}}}  \frac{1 - \Phi(y)}{\phi(y)} \\
& \ge \frac{1}{\exp\left(\frac{1}{6}\right) \sqrt{2 \pi}} \exp\left(-n D\left(\frac{1}{2}+\frac{k}{n} \middle\| \frac{1}{2}\right)\right) \frac{\pi}{(\pi-1)y+\sqrt{y^2+2 \pi}}. \qedhere
\end{align*}
\end{proof}

\section{Upper Bounds}
\label{section:upper-bounds}

We say that a random variable $X$ is \emph{$\sigma^2$-sub-Gaussian} (for some $\sigma \ge 0$) if
\begin{equation}
\label{equation:sigma-sub-gaussian}
\Exp \left[ e^{sX} \right] \le \exp\left( \frac{\sigma^2 s^2}{2} \right) \qquad \text{for all $s \in \R$} \; .
\end{equation}
It is straightforward to verify that $X \sim N(0,\sigma^2)$ is $\sigma^2$-sub-Gaussian. Indeed, for any $s \in \R$,
\begin{align*}
\Exp \left[ e^{sX} \right]
& = \int_{-\infty}^\infty \frac{1}{\sigma \sqrt{2\pi}} \exp\left( - \frac{x^2}{2\sigma^2} \right) e^{sx} dx \\
& = \exp\left( \frac{s^2\sigma^2}{2} \right) \int_{-\infty}^\infty \frac{1}{\sigma \sqrt{2\pi}} \exp\left( - \frac{(x - s\sigma^2)^2}{2\sigma^2} \right) dx \\
& = \exp\left( \frac{s^2\sigma^2}{2} \right) \; .
\end{align*}
We now show that a Rademacher random variable $Y$ (with distribution $\Pr[Y = +1] = \Pr[Y=-1] = \frac{1}{2}$)
is $1$-sub-Gaussian. Indeed, for any $s \in \R$,
\begin{align*}
\Exp \left[ e^{sY} \right]
= \frac{e^{s} + e^{-s}}{2}
= \frac{1}{2}\sum_{k=0}^\infty \frac{s^k}{k!}
+ \frac{1}{2}\sum_{k=0}^\infty (-1)^k \frac{s^k}{k!}
= \sum_{k=0}^\infty \frac{s^{2k}}{(2k)!}
\le \sum_{k=0}^\infty \frac{s^{2k}}{k! 2^k}
= \exp\left( \frac{s^2}{2} \right) \; .
\end{align*}
If $Y_1, Y_2, \dots, Y_n$ are independent $\sigma$-sub-Gaussian random variables, then $\sum_{i=1}^n Y_i$ is $(n\sigma^2)$-sub-Gaussian.
This follows from
$$
\Exp \left[ e^{s \sum_{i=1} Y_i} \right] = \prod_{i=1}^n \Exp \left[ e^{sY_i} \right] \; .
$$
This property proves that the symmetric random walk $Z^{(n)}$ of length $n$ is $n$-sub-Gaussian.

The upper bounds \eqref{equation:upper-bound-on-maximum-of-gaussians} and
\eqref{equation:upper-bound-on-maximum-of-random-walks} follow directly from
sub-Gaussianity of the variables involved and the following lemma.

\begin{lemma}[Maximum of sub-Gaussian random variables]
Let $X_1, X_2, \dots, X_d$ be (possibly dependent) $\sigma^2$-sub-Gaussian condition random variables.
Then,
$$
\Exp\left[ \max_{1 \le i \le d} X_i \right] \le \sigma \sqrt{2 \ln d} \; .
$$
\end{lemma}

\begin{proof}
For any $s > 0$, we have
\begin{align*}
\Exp \left[ \max_{1 \le i \le d} X_i \right]
& = \frac{1}{s} \Exp \left[ \max_{1 \le i \le d} \ln e^{s X_i} \right] \\
& \le \frac{1}{s} \ln \Exp \left[ \max_{1 \le i \le d} e^{s X_i} \right] \\
& \le \frac{1}{s} \ln \Exp \left[ \sum_{i=1}^d e^{s X_i} \right] \\
& = \frac{1}{s} \ln \sum_{i=1}^d \Exp \left[ e^{s X_i} \right] \\
& \le \frac{1}{s} \ln \left( d  \exp\left( \frac{\sigma^2 s^2}{2} \right) \right) \\
& = \frac{\ln d}{s} +  \frac{\sigma^2 s}{2} \; .
\end{align*}
Substituting $s=\frac{\sqrt{2 \ln d}}{\sigma}$ finishes the proof.
\end{proof}

\end{document}
