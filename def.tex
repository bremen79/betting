\section{Setting and Notation}
These papers deals with three different settings, hence we will fist describe the common notation and then each of one them separately, trying to show the similarities between them.

Real number variable and elements of arbitrary sets will be denoted by italic letters, e.g. $x,y$, while vectors will be denoted by bold letters, $\bw,\bu$.
Random variables will be indicated by capital italic letters, e.g. $X \sim N(0,\sigma^2)$.
A vector $\bx$ is a subgradient of a convex function $\ell$ at $\bv$ iff $\ell(\bu) - \ell(\bv) \ge \langle \bu - \bv, \bx \rangle$ for any $\bu$ in the domain of $\ell$. The differential set of $\ell$ at $\bv$, denoted by $\partial \ell(\bv)$, is the set of all the subgradients of $\ell$ at $\bv$.
We define the KL divergence between two Bernoulli distributions with parameters $p$ and $q$ as
\[
D(p||q) := p \log\frac{p}{q} + (1-p) \log\frac{1-p}{1-q}~.
\]
We will denote by $\fH$ an Hilbert space with inner product $\langle \cdot, \cdot\rangle$.

\vspace{0.2cm}\noindent\textbf{Binary and Continuous Coin Betting.}
The bettor starts with the amount of money $\epsilon$. 
At the end of each bet in the time step $t$ we denote the amount of money he has by $\wealth_t$ and by $\gain_t$ the amount of money gained through the $t$ bets. Hence, $\wealth_t=\gain_t+\epsilon$.
At each time step $t$, it has to bet a quantity of money $w_t$ equal to a fraction of his current money, $w_t:=b_t \, \wealth_{t-1}$ where $b_t \in (-1,1)$. Note that here we consider only betting strategies that will never result in a negative quantity of money owned. Hence, $|b_t|$ must be strictly less than 1, otherwise the algorithm could lose all the money in one round.
After the bet, the outcome of the coin $g_t \in \{-1,1\}$ is revealed and the bettor wins (or loses) the quantity $w_t g_t$.
Note that we will speak about a ``binary coin'' when the outcome of the coin is in $\{-1,1\}$, and ``continuous coin'' when $g_t \in [-1,1]$. The latter case model the situation in there are different possible prizes with unknown probabilities to be won. In both cases, we have $\wealth_t:=\gain_{t-1} + w_t g_t + \epsilon= \epsilon + \sum_{i=1}^t w_i g_i$, so that $\gain_0=0$ and $\wealth_0=\epsilon$.
We will extend to above definitions also to the the case that $\bw_t$ and $\bg_t$ belongs to a Hilbert space $\fH$ and $\norm{\bg_t}\leq 1$. Hence, with a slight abuse of notation we will also define $\gain_n=\sum_{t=1}^n \langle \bw_t, \bg_t \rangle$, $\wealth_t=\gain_t + \epsilon$ and $\bw_t = \bbeta_t \, \wealth_{t-1}$, with $\norm{\bbeta_t} < 1$.

\vspace{0.2cm}\noindent\textbf{\ac{OLO} and \ac{OCO}.}
Let $\fH$ be a Hilbert space. In the \ac{OLO} framework, at each round $t$ the algorithm receives a vector $\bg_t \in \fX$, picks a $\bw_t \in \fX \subset \fH$, and gains $\langle \bw_t,\bg_t \rangle$ (or equivalently loses -$\langle \bw_t,\bg_t \rangle$).
The aim of the algorithm is to minimize the \emph{regret}, that is the difference between the cumulative gains of the 
of an arbitrary and fixed competitor $\bu \in \fX$, $\sum_{t=1}^n \langle \bu,\bg_t \rangle$, and 
the cumulative gains of the algorithm, $\sum_{t=1}^n \langle \bw_t,\bg_t \rangle$.
In particular, define
\[
Regret_n(\bu) := \sum_{t=1}^n \langle \bg_t , \bu - \bw_t \rangle~.
\]
In the \ac{OCO} setting, instead, the algorithm receives convex loss functions $\ell_t$ rather than vectors.
Again, the aim is to minimize the regret, that in this case is defined as
\[
Regret^{OCO}_n(\bu) := \sum_{t=1}^n \left(\ell_t(\bw_t) -\ell_t(\bu)\right)~.
\]

\vspace{0.2cm}\noindent\textbf{Regularized ERM.}
Let $P$ a fixed but unknown distribution on $\fZ$, where $\fZ$ is an arbitrary set.
%Denote by $\frho(x):= \int_\fY y d\rho(y|x)$ the \emph{regression function}, where $\rho(\cdot|x)$ is the conditional probability measure at $x$ induced by $\rho$. 
%Denote by $\rho_\fX$ the marginal probability measure on $\fX$ and let $\Ltworho$ be the space of square integrable functions with respect to $\rho_\fX$, whose norm is denoted by $\normL{\f}:=\sqrt{\int_\fX \f^2(x) d \rho_\fX}$. Note that $\frho \in \Ltworho$.
We introduce a loss function  $\ell: \fH \times \fZ \rightarrow \R_+$, convex and \emph{$L$-Lipschitz} w.r.t. the first argument.
Note that this generalizes the case where we have the composition of a predictor parametrized by a vector $\bw$ and a loss function. Hence, for example, in the case of logistic regression define $\fZ= \fH \times [0,1]$ and $z=(\bx,\by)$ so that
$\ell(\bw,z)=\ln\left(1+\exp(- y \langle \bw, \bx\rangle)\right)$.
% We will consider \emph{$L$-Lipschitz} losses, that is $|\ell(x)-\ell(x')| \leq L |x-x'|$.
Define the \emph{$\ell$-risk} of $\bw \in \fH$, as $\RiskLoss(\bw):=\E_{Z \sim P} [\ell(\bw,Z)]$.
%Also, define $\flrho(x):=\argmin_{t \in \R} \int_\fY \ell(y t) d \rho(y|x)$, that gives the \emph{optimal $\ell$-risk}, $\RiskLoss(\flrho)=\inf_{\f \in \Ltworho} \RiskLoss(\f)$.
%, \ \forall x,x' \in \R$, and \emph{$H$-smooth} losses, that is differentiable losses with the first derivative $H$-Lipschitz. Note that a loss can be both Lipschitz and smooth.
Given a training set $\{z_t\}_{t=1}^n$ of samples drawn \ac{IID} from $P$, the regularized \ac{ERM} strategy finds a predictor $\hat{\bw}$ in a subset $\mathcal{S} \subseteq \fH$, such that
\[
\hat{f}= \argmin_{\bw \in \mathcal{S}} \ \lambda\, R(\bw) + \frac{1}{n} \sum_{t=1}^n \ell(\bw,z_t),
\]
where $R(\bw)$ is the regularizer.