\section{The Betting Algorithm for Continous Coins}
\label{sec:algo}

\begin{algorithm}[t]
  \begin{algorithmic}
  {
    \STATE{\bfseries Parameters:} $a>2,\delta>1$
    \STATE{\bfseries Initialize:} $\gain_0=\epsilon,\sg_0=\delta$
    \FOR{$t=1,2,\dots$}
    \STATE{Calculate fraction and direction to bet: $\beta_t=2 \, S\left(\frac{4 \theta_{t-1}}{a (\sg_{t-1} + 1)}\right)-1$}
    \STATE{Bet $w_t=\beta_t \gain_{t-1}$}
    \STATE{Win (or lose) $w_t g_t$}
    \STATE{Update your money: $\gain_t=\gain_{t-1}+w_t g_t$}
    \STATE{Update $\theta_t$: $\theta_t=\theta_{t-1}+g_t$}
    \STATE{Update $\sg_t$: $\sg_t=\sg_{t-1}+|g_t|$}
    \ENDFOR
  }
  \end{algorithmic}
  \caption{Continous Coin Betting (COCOB)}
  \label{alg:cocob}
\end{algorithm}

As said in Section~\ref{}, portfolio selection can be used to optimally solve betting and, as a consequence, stochastic optimization and self-tuning of regularized \ac{ERM}. However, no simple algorithm is known for portfolio selection.
Hence, in this section we show how a very simple algorithm can achieve a bound close to the optimal one.

The \ac{COCOB} algorithm is shown in Algorithm~\ref{alg:cocob} and we can prove the following guarantee for its performance.
%Without loss of generality, we will assume that $g_t$ are between $[-1,1]$.
%
\begin{theorem}
\label{theo:cocob}
Let $a\geq2$ and start with an amount of money equal to $\epsilon$. Bet at each round a quantity equal to
$\gain_{t-1} \left(2 \, S\left(\frac{4 \theta_{t-1}}{a (\sg_{t-1} + 1)}\right)-1\right)$, where $S:\R\rightarrow(0,1)$ and $S(x)=\frac{1}{1+exp(-x)}$ and receive the outcome $g_t \in [-1,1]$. Then
\begin{align*}
\gain_{n} 
\geq \epsilon \exp\left(\frac{\theta_{n}^2}{a \sg_{n}} - \sum_{i=1}^{n} \frac{|g_i|}{a( \sg_{i-1} + 1) } \right)
\geq \epsilon \exp\left(\frac{\theta_{n}^2}{a \sg_{n}} - \frac{1}{a} \ln \left(\sum_{t=1}^n \frac{|g_i|}{\delta}+1\right) \right)~.
\end{align*}
\end{theorem}
%
Notice that the $a$ regolates the trade-off between the first and second term. When $a=2$, we have a 